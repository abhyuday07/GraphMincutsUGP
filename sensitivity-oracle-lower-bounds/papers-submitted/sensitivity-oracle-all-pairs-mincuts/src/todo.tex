{\color{blue} To do list:
\begin{enumerate}
% \item What is the best deterministic bound on computing
% $(s,t)$-mincut in an undirected unweighted graph ? We need this information to show the nontriviality of the query time of our second data structure.\\

% Case 1: $m>nv$

% 1. $nm^{2/3}v^{1/6} > n^{5/3}v^{5/6}$, therefore, 
% $\frac{nv}{n^{5/3}v^{5/6}} \leq \frac{1}{n^{1/2}}$

% 2. $n^{3/2}m^{1/2} > n^{2}v^{1/2}$, therefore, 
% $\frac{nv}{n^{2}v^{1/2}} \leq \frac{1}{n^{1/2}}$

% Case 2: $m<nv$

% 1. $nm^{2/3}v^{1/6} > n^{5/6}m^{5/6}$, therefore, 
% $\frac{m}{n^{5/6}m^{5/6}} \leq \frac{1}{n^{1/2}}$

% 2. $\frac{m}{n^{3/2}m^{1/2}} \leq \frac{1}{n^{1/2}}$

% Therefore, it gives a $\sqrt{n}$ factor of improvement in query time, using the best algorithms.

\item @Abhyuday: Now I fully agree with you that if we are able to build our data structure(s) in $O(mF)$ time, it will add more value to our results. So we must strive to accomplish this goal and provide all details in Appendix.
\item What is the best lower bound that we can achieve on the fault tolerant all-pairs mincuts data structures for a graph with edge capacities polynomial in $n$ if the desired query time is $O(1)$ or $O($polylog$(n))$?
\item Distributed memory and parallelize query.
\item Time taken to report a mincut in 1 and 2.
\item Proof of E.1 uses the construction of strip corresponding to subbunch in 2.2. Try to refer it somehow.
\item Streamline Appendix G.
\item Add other fault tolerant data structures.
\item multigraph.
\end{enumerate}}


{

\color{brown} Doubts
\begin{enumerate}
    \item DKL figure issue.
    \item Resolve Strip Lemma (ii) part.
    \item Path $(\nu_1,\nu_2)$ and $(\nu_3,\nu_4)$ extendable intuition?
\end{enumerate}
}

% Suppose $u$ is a stretched unit in the flesh ${\cal F}_S$. Suppose ${\cal R}_1$ and ${\cal R}_2$ denote its two reachability cones. Each of ${\cal R}_1$,${\cal R}_2$,${\cal R}_1\setminus \{u\}$ and ${\cal R}_2 \setminus \{u\}$ define a steiner mincut. This can be proved by showing that no coherent path can interesect any of the above sets twice.

% Since ${\cal R}_1$ is a steiner mincut, it must correspond to some subbunch, say ${\cal B}$, in the skeleton ${\cal H}_S$. Suppose the path to which $u$ is mapped in the skeleton be $P(\nu_1,\nu_2)$. We claim that the subcactus rooted at $\nu_1$ wrt to path $P(\nu_1,\nu_2)$ is the minimal cut in the skeleton that represents this subbunch. The reason for the same is as follows. Suppose ${\cal H}_S({\cal B})\subset {\cal H}_S(\nu_1)$. Then $u$ must appear as a non-terminal in the strip representing this subbunch (because ${\cal R}_1\setminus \{u\}$ must be from the same subbunch). However ${\cal H}_S({\cal B})\subset {\cal H}_S(\nu_1)$, therefore $u$ cannot be mapped to any of the edges corresponding to cut ${\cal H}_S({\cal B})$. Thus, we arrive at a contradiction.

% \begin{lemma}
% If a stretched unit $u$ is mapped to path $P(\nu_1,\nu_2)$ in the skeleton, then the set of terminals in the reachability cones ${\cal R}_1$ and ${\cal R}_2$ coincide with the subbunch ${\cal H}_S(\nu_1)$ and ${\cal H}_S(\nu_2)$.
% \end{lemma}

% Now, we can easily derive Lemma 17. Suppose $u'$ be any stretched unit, such that $u'\in {\cal R}_1$. Moreover, let ${\cal R}_1'$ and ${\cal R}_2'$ denote the reachability cones of $u'$ such that ${\cal R}_1'\subseteq {\cal R}_1$ and ${\cal R}_2 \subseteq {\cal R}_2'$. Observe that this implies that the $\nu_3 \in {\cal H}_S(\nu_1)$ and $\nu_2 \in {\cal H}_S(\nu_4)$.