\begin{table}
\centering
\begin{tabular}{llll}
\textcolor[rgb]{0.149,0.196,0.22}{Strength}                                                                                                                                                                                                                                                                                                                                                                                                                                                                                                                                                                                                                                                                                                                                                                                                                                                                                                                                                                                                                                                                                                                 & \textcolor[rgb]{0.149,0.196,0.22}{Weakness}                                                                                                                                                                                                                                                                                                                                                                                                                                                                                                                                                                                                                                                                                                                                                                                                                                                                                                                                                                                                                                                                                                                                                                                                                                                                                                                               &  &   \\
\begin{tabular}[c]{@{}l@{}}\textcolor[rgb]{0.149,0.196,0.22}{It is the first result regarding the one-failure structure}\\\textcolor[rgb]{0.149,0.196,0.22}{for all-pair s-t min-cut.}\\\textcolor[rgb]{0.149,0.196,0.22}{}\end{tabular}                                                                                                                                                                                                                                                                                                                                                                                                                                                                                                                                                                                                                                                                                                                                                                                                                                                                                                                    & \begin{tabular}[c]{@{}l@{}}\textcolor[rgb]{0.149,0.196,0.22}{The major part of this paper is the description of}\\\textcolor[rgb]{0.149,0.196,0.22}{the linear space structure, but the query time of this structure is not so}\\\textcolor[rgb]{0.149,0.196,0.22}{efficient.}\\\\\textcolor[rgb]{0.149,0.196,0.22}{This paper is very well written. Maybe it will be better to move the O(n}\textcolor[rgb]{0.149,0.196,0.22}{2}\textcolor[rgb]{0.149,0.196,0.22}{)-space}\\\textcolor[rgb]{0.149,0.196,0.22}{and O(1)-query time structure in front since it is easier to comprehend.}\\\textcolor[rgb]{0.149,0.196,0.22}{}\end{tabular}                                                                                                                                                                                                                                                                                                                                                                                                                                                                                                                                                                                                                                                                                                                                &  &   \\
\begin{tabular}[c]{@{}l@{}}\textcolor[rgb]{0.149,0.196,0.22}{The paper makes progress on a fundamental question in algorithm design. While we}\\\textcolor[rgb]{0.149,0.196,0.22}{have witnessed a substantial number of works on fault-tolerant distance-based graph}\\\textcolor[rgb]{0.149,0.196,0.22}{approximations, not much was known about the higher-connectivity structure of graphs}\\\textcolor[rgb]{0.149,0.196,0.22}{in this setting.}\\\textcolor[rgb]{0.149,0.196,0.22}{Even though the techniques used in this paper may not feel very novel to someone}\\\textcolor[rgb]{0.149,0.196,0.22}{being fairly familiar with fault-tolerant data-structures and network flow theory,}\\\textcolor[rgb]{0.149,0.196,0.22}{the paper certainly does some non-trivial chunk of work to bring together the}\\\textcolor[rgb]{0.149,0.196,0.22}{Steiner min-cut representations and labelling schemes. Moreover, these}\\\textcolor[rgb]{0.149,0.196,0.22}{data-structures themselves are far from being easy to grasp.}\\\textcolor[rgb]{0.149,0.196,0.22}{The writing is clear and to the point.}\\\textcolor[rgb]{0.149,0.196,0.22}{}\end{tabular} & \begin{tabular}[c]{@{}l@{}}\textcolor[rgb]{0.149,0.196,0.22}{One obvious limitation is that the current data-structure supports only 1 edge}\\\textcolor[rgb]{0.149,0.196,0.22}{deletion. The fault-tolerant setting is considered to be more competitive when the data-structure supports up to K-edge deletions. This would be more exciting, as it is closer to maintaining all-pairs (s-t) min-cuts dynamically.}\\\textcolor[rgb]{0.149,0.196,0.22}{I find it strange that the paper does not discuss the preprocessing time of the}\\\textcolor[rgb]{0.149,0.196,0.22}{data-structure and only focuses on space complexity. I would assume that running}\\\textcolor[rgb]{0.149,0.196,0.22}{time is a more relevant parameter for this problem.~}\\\textcolor[rgb]{0.149,0.196,0.22}{Another concern regarding the paper is that in various settings this paper is only}\\\textcolor[rgb]{0.149,0.196,0.22}{providing query complexity improvements (of more than logarithmic factors) over}\\\textcolor[rgb]{0.149,0.196,0.22}{deterministic algorithms. In these cases, randomized solutions with the same query complexity (up to logarithmic factors) could be achieved just by applying}\\\textcolor[rgb]{0.149,0.196,0.22}{state-of-the-art randomized maximum flow / minimum s-t cut algorithms blackbox.}\\\textcolor[rgb]{0.149,0.196,0.22}{}\end{tabular} &  &   \\
\begin{tabular}[c]{@{}l@{}}\textcolor[rgb]{0.149,0.196,0.22}{The authors study a new problem of fault-tolerant all-pairs mincuts.}\\\textcolor[rgb]{0.149,0.196,0.22}{The authors develop two data-structures, one is query optimal (constant query time)}\\\textcolor[rgb]{0.149,0.196,0.22}{using O(n\^2) space, and the other is space optimal - O(m) space with a query}\\\textcolor[rgb]{0.149,0.196,0.22}{a procedure that is at least \textbackslash{}sqrt\{n\} times more efficient than the static algorithm.}\\\textcolor[rgb]{0.149,0.196,0.22}{The data-structures developed by the authors are interesting, innovative, and give a solid contribution to this field.}\\\textcolor[rgb]{0.149,0.196,0.22}{The main strength of this paper, in my opinion, is researching a new interesting}\\\textcolor[rgb]{0.149,0.196,0.22}{question and giving a sophisticated and efficient data-structure for this problem.}\\\textcolor[rgb]{0.149,0.196,0.22}{}\end{tabular}                                                                                                                                                                            & \begin{tabular}[c]{@{}l@{}}\textcolor[rgb]{0.149,0.196,0.22}{As there is no previous work on fault-tolerant mincuts, it is difficult to evaluate}\\\textcolor[rgb]{0.149,0.196,0.22}{the achievement of how efficient the query time is and how further can the query}\\\textcolor[rgb]{0.149,0.196,0.22}{time be improved. Is there any relevant conditional lower bound?}\\\textcolor[rgb]{0.149,0.196,0.22}{~}\textcolor[rgb]{0.149,0.196,0.22}{I think that the authors study an interesting problem, present a sophisticated and}\\\textcolor[rgb]{0.149,0.196,0.22}{clever solution, and this work may lead to additional research of fault-tolerant}\\\textcolor[rgb]{0.149,0.196,0.22}{mincuts. Also, the writeup is very good.~ Thus, I tend towards acceptance, with the}\\\textcolor[rgb]{0.149,0.196,0.22}{weakness that it is hard to evaluate their bounds without lower bounds of any kind.}\\\textcolor[rgb]{0.149,0.196,0.22}{}\end{tabular}                                                                                                                                                                                                                                                                                                                                                                                                             &  &  
\end{tabular}
\end{table}