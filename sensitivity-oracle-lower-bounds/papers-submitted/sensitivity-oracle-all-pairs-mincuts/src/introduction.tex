Graph mincut is a fundamental structure in graph theory with numerous applications. Let $G=(V,E)$ be an undirected unweighted connected graph on $n=|V|$ vertices and $m=|E|$ edges.
Two most common types of mincuts are global mincuts and pairwise mincuts. A set of edges with the least cardinality whose removal disconnects the graph is called a global mincut. For any pair of vertices $s,t\in V$, 
a set of edges with the least cardinality whose removal disconnects $t$ from $s$ is called a pairwise mincut for $s,t$ or simply a $(s,t)$-mincut. A more general notion is that of Steiner mincuts. For any given set $S\subseteq V$ of vertices, a set of edges with the least cardinality whose removal disconnects $S$ is called a Steiner mincut for $S$. It is easy to observe that the Steiner mincuts for $S=V$ are the global mincuts and for $S=\{s,t\}$ are $(s,t)$-mincuts.

% {\color{red} (-)
% While designing an algorithm for a graph problem, one usually assumes that the underlying graph is static. But, this assumption is unrealistic for most of the real-world graphs where vertices and/or edges do fail, though occasionally. While it is impractical to assume that a real-world graph is immune to such failures, it is also a fact that these failures are transient - an edge (or a vertex) once failed, becomes active after some time due to the simultaneous repair mechanism that is undertaken in most of the real-world graphs. So the set of failed edges (or vertices), though small in size, keeps changing with time. Therefore, a more realistic way to model a real-world graph is to assume that, at any time, there will be at most $k$ failed vertices or edges for some $k$ defined suitably. Solving a graph problem in this model requires building a compact data structure such that given any set of at most $k$ failed vertices or edges, the solution of the problem can be reported efficiently.
% }

 Typically a graph algorithmic problem is solved assuming that the underlying graph is static. However, changes are inevitable in most real world graphs. For example, real world graphs are prone to failures of edges. These failures are transient in nature. As a result, the set of failed edges at any time, though small in size, may keep changing as time goes by. Therefore, a data structure for a problem on a real world graph has to consider a given set of failed edges as well to answer a query correctly. It is also important to efficiently know the {\em change} in the solution of the problem for a given set of failed edges or a given set of new edges. This knowledge can help to efficiently determine the {\em impact} of the failure/insertion of an edge on the solution of the problem. Solving a graph problem in this model requires building a compact data structure such that given any set of at most $k$ changes, i.e. failures (or insertions) of edges (or vertices), the solution of the problem can be reported efficiently. It is important to note that this model subsumes the fault-tolerant model, in which we are only concerned about failure of edges (or vertices).

%{\color{blue}
%While designing an algorithm for a graph problem, one usually assumes that the underlying graph is static. But, this assumption is unrealistic for most of the real-world graphs where vertices and/or edges suffer change, though occasionally. While it is impractical to assume that real-world graphs are immune to changes, it this also a fact that many-a-times these changes are transient - once addition (or failure) of an edge (or a vertex) takes place, it may get reverted back to its initial state. Also, in many circumstances the time between consecutive changes is large enough to preprocess the data structure again. Therefore, a more realistic way to model a real-world graph is to assume that, at any time, there will be at most $k$ changes to edges (or vertices) for some $k$ defined suitably. Solving a graph problem in this model requires building a compact data structure such that given any set of at most $k$ changes, i.e. additions (or failures) of edges (or vertices), the solution of the problem can be reported efficiently. It is important to note that this model subsumes the fault-tolerant model, in which we are only concerned about failure of edges (or vertices).
%}


% {\color{red} (-)
% In the past, many elegant fault-tolerant algorithms have been designed for various classical problems, namely,  connectivity \cite{DBLP:journals/algorithmica/FrigioniI00,DBLP:journals/siamcomp/ChanPR11,DBLP:conf/soda/DuanP17}, shortest-paths \cite{DBLP:conf/stoc/BernsteinK09,DBLP:journals/siamcomp/DemetrescuTCR08,DBLP:conf/stoc/ChechikC20}, graph spanners \cite{DBLP:journals/siamcomp/ChechikLPR10,DBLP:journals/tcs/BraunschvigCPS15}, SCC \cite{DBLP:journals/algorithmica/BaswanaCR19} , DFS tree \cite{DBLP:journals/siamcomp/BaswanaCC019}, and BFS structure \cite{DBLP:journals/talg/ParterP16,DBLP:journals/talg/ParterP18}. However, little is known about fault-tolerant data structures for various types of mincuts. The problem of fault-tolerant all-pairs mincuts aims at preprocessing a given graph to build a compact data structure so that the following query can be answered efficiently for any $s,t\in V$ and $(x,y)\in E$.
% }

%{\color{blue}
In the past, many elegant fault-tolerant data structures have been designed for various classical problems, namely,  connectivity \cite{DBLP:journals/algorithmica/FrigioniI00,DBLP:journals/siamcomp/ChanPR11,DBLP:conf/soda/DuanP17}, shortest-paths \cite{DBLP:conf/stoc/BernsteinK09,DBLP:journals/siamcomp/DemetrescuTCR08,DBLP:conf/stoc/ChechikC20}, graph spanners \cite{DBLP:journals/siamcomp/ChechikLPR10,DBLP:journals/tcs/BraunschvigCPS15}, SCC \cite{DBLP:journals/algorithmica/BaswanaCR19} , DFS tree \cite{DBLP:journals/siamcomp/BaswanaCC019}, and BFS structure \cite{DBLP:journals/talg/ParterP16,DBLP:journals/talg/ParterP18}. Recently, a data structure was designed which could report the mincut between a pair of vertices amidst insertion of an edge \cite{DBLP:conf/esa/BaswanaGK20}. However, little is known about fault-tolerant data structures for various types of mincuts.

The problem of sensitivity oracle for all-pairs mincuts aims at preprocessing a given graph to build a compact data structure so that the following queries can be answered efficiently for any $s,t\in V$ and $(x,y) \in V \times V$.

\begin{itemize}
    \item {\textsc{ft-Mincut}}$(s,t,x,y)$: Report a $(s,t)$-mincut in $G$ after the failure of edge $(x,y)\in E$.
    \item {\textsc{in-Mincut}}$(s,t,x,y)$: Report a $(s,t)$-mincut in $G$ upon insertion of edge $(x,y)\in V \times V$.
\end{itemize}
%}

% \noindent
% FT-mincut$(u,v,x,y)$: Report the value of $(u,v)$-mincut in $G$ after the failure/removal of edge $(x,y)$, if exists, from $E$.\\
% \noindent
% {\textsc{ft-mincut}}$(s,t,x,y)$: Report a $(s,t)$-mincut in $G$ after the failure of edge $(x,y)$, if exists, from $E$.\\



% Upon failure of edge $(x,y)$, the value of $(s,t)$-mincut for any $s,t\in V$ either decreases by unity or remains unchanged. We now state the necessary and sufficient condition for the value of $(s,t)$-mincut to decrease upon failure of edge $(x,y)$.



% It follows from Fact \ref{fact:(x,y)-lies-in-(s,t)-mincut} that any {\textsc{ft-mincut}}($s,t,x,y$) query can be answered by determining whether $(x,y)$ belongs to any $(s,t)$-mincut. Unfortunately, there can be exponential number of $(s,t)$-mincuts in a given graph \cite{DBLP:journals/mp/PicardQ82}. Thus, designing a compact data structure to answer this query efficiently turns out to be a challenging task.



\paragraph{Previous Results:} 
\label{sec:previous-results}

There exists a classical ${\cal O}(n)$ size data structure that stores all-pairs mincuts \cite{GH61} known as Gomory-Hu tree. It is a tree on the vertex set $V$ that compactly stores a mincut between each pair of vertices. However, we cannot determine using a Gomory-Hu tree whether the failure of an edge will affect the $(s,t)$-mincut unless this edge belongs to the $(s,t)$-mincut present in the tree. We can get a fault-tolerant data structure by storing $m$ Gomory-Hu trees, one for each edge failure. The overall data structure occupies ${\cal O}(mn)$ space and takes ${\cal O}(1)$ time to report the value of $(s,t)$-mincut for any $s,t\in V$ upon failure of a given edge. 
% However, the ${\cal O}(mn)$ space occupied by this data structure is far from the size of the graph. 
% To the best of our knowledge, even after $60$ years of existence of Gomory-Hu tree, there exists no data structure for this problem which takes $o(mn)$ space and a non-trivial query time.
%{ \color{blue}
Recently, for the case of single edge insertion only, Baswana, Gupta, and Knollman \cite{DBLP:conf/esa/BaswanaGK20} designed an ${\cal O}(n^2)$ data structure that can efficiently report the value of a $(s,t)$-mincut upon insertion of an edge $(x,y)$ for any given $s,t,x,y \in V$.  To the best of our knowledge, there exists no sensitivity data structure for all-pairs mincuts problem which takes $o(mn)$ space and has a non-trivial query time.
%}


\paragraph{Our Contribution:} We present two sensitivity data structures for the all-pairs mincuts problem in an undirected unweighted graph.

\begin{enumerate}
    \item 
%{\color{blue}
  Our first data structure occupies ${\cal O}(n^2)$ space and guarantees ${\cal O}(1)$ query time to report the value of resulting $(s,t)$-mincut for any $s,t \in V$ upon failure/insertion of any edge. Moreover, the set of vertices defining a resulting $(s,t)$-mincut after the update can be reported in ${\cal O}(n)$ time which is worst-case optimal. Our data structure also subsumes the previous results of Baswana, Gupta, and Knollman \cite{DBLP:conf/esa/BaswanaGK20} which only handles an edge insertion. An interesting byproduct of our data structure, which is of independent interest, is that it can output a compact representation \cite{DBLP:journals/mp/PicardQ82} storing all $(s,t)$-mincuts for any $s,t \in V$ in ${\cal O}(m)$ time which is optimal (see Appendix \ref{appendix:xxx}).
%}

    \item
Our second data structure occupies ${\cal O}(m)$ space.
% and is indeed optimal (Lemma \ref{lem:lower-bound})
The query time guaranteed by this data structure to report the value of $(s,t)$-mincut for any $s,t\in V$ upon failure/insertion of any edge is ${\cal O}(\min(m,nc_{s,t}))$ where $c_{s,t}$ is the value of $(s,t)$-mincut in graph $G$. This query time 
is faster by a factor of $\Omega(\min(m^{1/3},\sqrt{n}))$ (details in Appendix \ref{appendix:non-trivial-query-time}) compared to the best known deterministic algorithm \cite{DBLP:conf/focs/GoldbergR97a,DBLP:conf/stoc/KargerL98,DBLP:journals/corr/abs-2003-08929} to compute a $(s,t)$-mincut from scratch.

\end{enumerate}
\begin{remark}
There exists a Las-Vegas algorithm \cite{DBLP:conf/stoc/KargerL98} to compute static $(s,t)$-mincut in ${\cal O}(\max(m,nc_{s,t}))$ time. 
% However, it remains an open problem if there exists a deterministic algorithm for static $(s,t)$-mincut with ${\cal O}(m^{1+o(1)})$ time.
However, no deterministic algorithm till date matches this bound even within the factor of $m^{o(1)}$ \cite{DBLP:conf/focs/GoldbergR97a,DBLP:conf/stoc/KargerL98,DBLP:journals/corr/abs-2003-08929}.
\end{remark}

Our first data structure achieves optimal query time but uses ${\cal O}(n^2)$ space. The second data structure, on the other hand, takes linear space but has ${\cal O}(\min(m,nc_{s,t}))$ query time. It remains an open problem if there exists a data structure that can achieve a space-time trade-off. Interestingly, this problem also remains open for much studied static all-pairs directed reachability problem  \cite{DBLP:journals/siamcomp/Patrascu11,DBLP:conf/wads/GoldsteinKLP17}.
% {\color{red} Check if this problem is open for APSP (undirected) as well.}

% Similar problem also remains open \cite{DBLP:journals/siamcomp/Patrascu11,DBLP:conf/wads/GoldsteinKLP17} for the much studied static all-pair reachability problem in directed graphs. 

% {\color{red} We show that this trade-off conditionally holds at least for any sensitivity-oracle that can handle dual insertions/failures.}

% It is worthwhile to mention that Karger and Levine \cite{DBLP:conf/stoc/KargerL02} gave an $\tilde{\cal O}(\max(m,nc_{s,t}))$ time Las-Vegas algorithm for computing maxflow between $s$ and $t$ and thereby an $(s,t)$-mincut. However, since our query algorithm works in deterministic framework we do not intend to compare the query time of our data structure with Karger and Levine's algorithm. {\color{red} The previous statement appears to show that we are indifferent towards randomized results. May be, we can use some better statements.~Also can we not compare ${\cal O}(\min(m,nc_{s,t}))$ and
% $\tilde{\cal O}(\max(m,nc_{s,t}))$ }


% Both these data structures can also report the resulting $(s,t)$-mincut incorporating the failure in ${\cal O}(\min(m,n c_{s,t}))$ time.

% { \color{blue}
In order to design our data structures, we present an efficient solution for a related problem of independent interest, called edge-containment query on a mincut defined as follows.\\
%}

\noindent
% {\textsc{Edge-Contained}}$(s,t,(x,y))$: Check if an edge $(x,y)\in E$ belong to some $(s,t)$-mincut.\\
{\textsc{Edge-Contained}}$(s,t,E_y)$: Check if a given set of edges $E_y\subset E$ sharing a common endpoint $y$ belong to some $(s,t)$-mincut.\\


Using Fact \ref{fact:(x,y)-lies-in-(s,t)-mincut}, any fault-tolerant query can be answered using old value of $(s,t)$-mincut and edge-containment query on $(s,t)$-mincut for $E_y = \{(x,y)\}$. Fact \ref{fact:(x,y)-insertion} helps us in answering the query of $(s,t)$-mincut upon the insertion of any edge.


%  Upon failure of edge $(x,y)$, the value of $(s,t)$-mincut for any $s,t\in V$ either decreases by unity or remains unchanged. We now state the necessary and sufficient condition for the value of $(s,t)$-mincut to decrease upon failure of edge $(x,y)$.
 
 \begin{fact}
\label{fact:(x,y)-lies-in-(s,t)-mincut}
The value of $(s,t)$-mincut decreases (by unity) on failure of an edge $(x,y)$ if and only if $(x,y)$ lies in \textit{at least one} $(s,t)$-mincut.
\end{fact}

 \begin{fact}
\label{fact:(x,y)-insertion}
The value of $(s,t)$-mincut increases (by unity) on insertion of an edge $(x,y)$ if and only if $x$ and $y$ are separated in \textit{all} $(s,t)$-mincuts.
\end{fact}
% Using a data structure for this problem, we can answer a fault-tolerant query upon failure of edge $(x,y)$ as follows. We perform the corresponding edge-containment query for $E_y = \{(x,y)\}$. The value of $(s,t)$-mincut will reduce by unity if the edge-containment query evaluates to true. We can keep a Gomory-Hu tree of ${\cal O}(n)$ size as an auxiliary data structure to lookup the old value of $c_{s,t}$. Our first data structure can answer edge-containment queries for $|E_y|=1$ in  ${\cal O}(1)$ time. On the other hand, our second data structure works for any given set $E_y$ but takes ${\cal O}(\min(m,nc_{s,t}))$ time. 


% \begin{remark}
% It is worthwhile to note that edge-containment queries cannot be extended to handle more than one edge failure. This is because Fact \ref{fact:(x,y)-lies-in-(s,t)-mincut} doesn't hold for multiple edges failures.
% \end{remark}




\paragraph{Related Work:} A related problem is that of maintaining mincuts in a dynamic environment. Until recently, most of the work on this problem has been limited to global mincuts. Thorup \cite{DBLP:journals/combinatorica/Thorup07} gave a Monte-Carlo algorithm for maintaining a global mincut of polylogarithmic size with ${\tilde {\cal O}}(\sqrt{n})$ update time.  He also showed how to maintain a global mincut of arbitrary size with $1+o(1)$-approximation within the same time-bound. Goranci, Henzinger and Thorup \cite{DBLP:journals/talg/GoranciHT18} gave a deterministic incremental algorithm for maintaining a global mincut with amortized ${\tilde{\cal O}}(1)$ update time and ${\cal O}(1)$ query time. Hartmann and Wagner \cite{DBLP:conf/isaac/HartmannW12} designed a fully dynamic algorithm for maintaining all-pairs mincuts  which provided significant speedup in many real-world graphs, however, its worst-case asymptotic time complexity is not better than the best static algorithm for an all-pairs mincut tree. Recently, there is a fully-dynamic algorithm \cite{DBLP:journals/corr/abs-2005-02368} that approximates all-pairs mincuts up to a nearly logarithmic factor in ${\tilde{\cal O}}(n^{2/3} )$ amortized time against an oblivious adversary, and ${\tilde{\cal O}}(m^{3/4} )$ time against an adaptive adversary. To the best of our knowledge, there exists no non-trivial dynamic algorithm for all-pairs \textit{exact} mincut. We feel that the insights developed in our paper may be helpful for this problem.

\noindent

% {
%     \color{blue}
%     \begin{itemize}
%         \item First para remains intact.
%         \item Only two components can be effectively used to answer ft queries for Steiner set.
%         \item Draw analogy between flesh and dag. How topological ordering can be used to efficiently report ft-mincut in a dag. Similar idea for flesh.
%         \item Additional augmentation to projection mapping can help report mincut efficiently.
%     \end{itemize}

% }

\paragraph{Overview of our results:} Dinitz and Vainshtein \cite{DBLP:conf/stoc/DinitzV94,DBLP:journals/siamcomp/DinitzV00}
presented a novel data structure called {\em connectivity carcass} that stores all Steiner mincuts for a given Steiner set $S\subseteq V$ in ${\cal O}(\min(m,nc_S))$ space, where $c_S$ is the value of the Steiner mincut. %We observe that this data structure can be used easily to design a fault-tolerant data structure for all-pairs mincuts that occupies $O(mn)$-space. 
Katz, Katz, Korman and Peleg \cite{DBLP:journals/siamcomp/KatzKKP04} presented a data structure of ${\cal O}(n)$ size for labeling scheme of all-pairs mincuts. This structure hierarchically partitions the vertices based on their connectivity in the form of a rooted tree. In this tree, each leaf node is a vertex in set $V$ and each internal node $\nu$ stores the Steiner mincut value of the set $S(\nu)$ of leaf nodes in the subtree rooted at $\nu$. %stores all-pairs mincut value.
% Apparently, both these data structures seem to be designed for a static graph. 
We observe that if each internal node $\nu$ of the hierarchy tree ${\cal T}$
is augmented with the connectivity carcass of $S(\nu)$, we get a data structure for the edge-containment query. This data structure occupies ${\cal O}(mn)$ space. An edge-containment query can be answered using the connectivity carcass at the Lowest Common Ancestor (LCA) of the given pair of vertices. However, this data structure will require ${\cal O}(m)$ time to report the mincut between the given pair of vertices upon failure of an edge. As mentioned in previous results, storing $m$ copies of Gomory-Hu tree trivially is a better data structure for this problem.

We focus our attention on a fixed $s,t$ pair. There exists a dag structure \cite{DBLP:journals/mp/PicardQ82,DBLP:journals/siamcomp/DinitzV00} that stores all $(s,t)$-mincuts. Exploiting the acyclic structure of the dag and the transversality of each $(s,t)$-mincut in this DAG, we design a data structure that occupies just ${\cal O}(n)$ space. Not only it can report the value of $(s,t)$-mincut after failure of an edge in ${\cal O}(1)$ time but also a resulting $(s,t)$-mincut in ${\cal O}(n)$ time. We can trivially store this data structure for all possible $(s,t)$ pairs to get a sensitivity oracle for all-pairs mincuts. However, the ${\cal O}(n^3)$ space taken by this data structure is worse than the trivial structure mentioned in previous results.


% {\color{red}
% Can we concisely convey what is topological ordering in this multi-terminal dag-like structure? Even vaguely. We are not storing multi-terminal dag-like structure.
% }

In their seminal work, Dinitz and Vainshtein \cite{DBLP:conf/stoc/DinitzV94,DBLP:journals/siamcomp/DinitzV00} also showed that there exists a multi-terminal dag-like structure that implicitly stores all Steiner mincuts. This structure is called \textit{flesh} and takes the majority of the space in the connectivity carcass. Now, the question arises if we can exploit the acyclicity and transversality of Steiner mincuts just like the way we did it for $(s,t)$-mincuts, to get a space efficient data structure for the problem. We show that the concepts analogous to topological ordering can be extended to the flesh as well. Moreover, we observe that only a part of connectivity carcass, avoiding the flesh, is required for our query. This part can be suitably augmented to get a data structure that can report a $(s,t)$-mincut upon failure of an edge if $s$ and $t$ are Steiner vertices separated by some Steiner mincut. This data structure takes only ${\cal O}(n)$ space and can report the value of $(s,t)$-mincut and a resulting such cut in ${\cal O}(1)$ and ${\cal O}(n)$ time respectively. By augmenting each internal node of the hierarchy tree ${\cal T}$ with this data structure, we get our ${\cal O}(n^2)$ space sensitivity oracle.

As we move down the hierarchy tree, the size of Steiner set associated with an internal node reduces. So, to make the data structure more compact, a possible approach is to associate a smaller graph $G_\nu$ for each internal node $\nu$ that is {\em small enough} to improve the overall space-bound, yet {\em large enough} to retain the internal connectivity of set $S(\nu)$. 
However, such a compact graph cannot directly answer the edge-containment query as it does not even contain the information about all edges in $G$. A possible way to overcome this challenge is to transform any edge-containment query in graph $G$ to an equivalent query in graph $G_{\nu}$. We show that not only such a transformation exists, but it can also be computed efficiently. We model the query transformation as a multi-step procedure. 
% The following result captures a single step of this procedure. 
We now provide the crucial insight that captures a single step of this procedure.

% The key observation that allows us to do query transformation is as follows. 
Given an undirected graph $G=(V,E)$ and a Steiner set $S\subseteq V$, let $S'\subset S$ be any maximal set with connectivity strictly greater than that of $S$. We can build a quotient graph $G_{S'}=(V_{S'},E_{S'})$ such that $S' \subset V_{S'}$ with the following property.

{\em
For any two vertices $s,t \in S'$ and any set of edges $E_y$ incident on vertex $y$ in $G$, there exists a set of edges $E_{y'}$ incident on a vertex $y'$ in $G_{S'}$ such that $E_y$ lies in a $(s,t)$-mincut in $G$ if and only if $E_{y'}$ lies in a $(s,t)$-mincut in $G_{S'}$. 
}

We build graph $G_\mu$ associated with each internal node $\mu$ of hierarchy tree ${\cal T}$ as follows. For the root node $r$, $G_r = G$. For any other internal node $\mu$, we use the above property to build the graph $G_{\mu}$ from $G_{\mu'}$, where $\mu'$ is the parent of $\mu$. 
Our ${\cal O}(m)$ space data-structure is the hierarchy tree ${\cal T}$ where each internal node $\mu$ is augmented with the connectivity carcass for $G_\mu$ and the Steiner set $S(\mu)$.
% A high-level description of our query algorithm is as follows. We traverse the path from the root node to the LCA of $s$ and $t$. We keep transforming the edge-containment query for each edge in this path. At the LCA of $s$ and $t$, we stop and perform the query using the connectivity carcass stored at this node. Following a rigorous analysis, we show that this data structure takes only ${\cal O}(m)$ space and can answer any edge-containment query in ${\cal O}(\min(m,nc_{s,t}))$ time.

\paragraph{Organization of the paper:}

In addition to the basic preliminaries, Section \ref{sec:prelimiaries} presents compact representation for various mincuts. Section \ref{sec:n^2-space-sensitivity-oracle} gives the details of the ${\cal O}(n^2)$ space sensitivity oracle. Section \ref{sec:query-transformation} gives insights into $3$-vertex mincuts that form the foundation for transforming an edge-containment query in original graph to a compact graph. Section \ref{sec:insights-nearest-mincuts} builds tools for handling edge-insertion in linear space data structure. We give the construction of compact graph for query transformation in Section \ref{sec:compact-graph-section}. Using this compact graph as a building block, we present the linear space sensitivity data structure in Section \ref{sec:final-ds}.