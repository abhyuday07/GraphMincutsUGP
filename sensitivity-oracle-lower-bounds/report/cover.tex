% -*-latex-*-
% 
% For questions, comments, concerns or complaints:
% thesis@mit.edu
% 
%
% $Log: cover.tex,v $
% Revision 1.9  2019/08/06 14:18:15  cmalin
% Replaced sample content with non-specific text.
%
% Revision 1.8  2008/05/13 15:02:15  jdreed
% Degree month is June, not May.  Added note about prevdegrees.
% Arthur Smith's title updated
%
% Revision 1.7  2001/02/08 18:53:16  boojum
% changed some \newpages to \cleardoublepages
%
% Revision 1.6  1999/10/21 14:49:31  boojum
% changed comment referring to documentstyle
%
% Revision 1.5  1999/10/21 14:39:04  boojum
% *** empty log message ***
%
% Revision 1.4  1997/04/18  17:54:10  othomas
% added page numbers on abstract and cover, and made 1 abstract
% page the default rather than 2.  (anne hunter tells me this
% is the new institute standard.)
%
% Revision 1.4  1997/04/18  17:54:10  othomas
% added page numbers on abstract and cover, and made 1 abstract
% page the default rather than 2.  (anne hunter tells me this
% is the new institute standard.)
%
% Revision 1.3  93/05/17  17:06:29  starflt
% Added acknowledgements section (suggested by tompalka)
% 
% Revision 1.2  92/04/22  13:13:13  epeisach
% Fixes for 1991 course 6 requirements
% Phrase "and to grant others the right to do so" has been added to 
% permission clause
% Second copy of abstract is not counted as separate pages so numbering works
% out
% 
% Revision 1.1  92/04/22  13:08:20  epeisach

% NOTE:
% These templates make an effort to conform to the MIT Thesis specifications,
% however the specifications can change. We recommend that you verify the
% layout of your title page with your thesis advisor and/or the MIT 
% Libraries before printing your final copy.

\title{Sensitivity Oracle for All-Pairs Mincuts}

\author{Abhyuday Pandey\\(BT/CSE/170039)\\\\\\\textbf{Supervisor:} Dr. Surender Baswana}
\date{\today}

% \maketitle
% \begin{center}
%     \includegraphics[]{templates/iitk_logo.jpg}
% \end{center}

\makeatletter
    \begin{titlepage}
        \begin{center}
            {\huge \bfseries  \@title }\\[4ex] 
            {\large  Abhyuday Pandey}\\[4ex]
            {\large BT/CSE/170039}\\[4ex]
            {\large \textbf{Supervisor:} Dr. Surender Baswana}\\[20ex]
            \includegraphics[height=5cm]{templates/iitk_logo.jpg}\\[20ex] 
            
            {\large \@date}
        \end{center}
    \end{titlepage}
\makeatother
\thispagestyle{empty}
\newpage

% The abstractpage environment sets up everything on the page except
% the text itself.  The title and other header material are put at the
% top of the page, and the supervisors are listed at the bottom.  A
% new page is begun both before and after.  Of course, an abstract may
% be more than one page itself.  If you need more control over the
% format of the page, you can use the abstract environment, which puts
% the word "Abstract" at the beginning and single spaces its text.

%% You can either \input (*not* \include) your abstract file, or you can put
%% the text of the abstract directly between the \begin{abstractpage} and
%% \end{abstractpage} commands.

% First copy: start a new page, and save the page number.
% Uncomment the next line if you do NOT want a page number on your
% abstract and acknowledgments pages.
% \pagestyle{empty}
\section*{Abstract}
\subfile{abstract}
\vfill
% {These results are submitted to the $53^{rd}$ Annual ACM Symposium on Theory of Computing
% June $21-25, 2021$ in Rome, Italy.}

\pagebreak
% Additional copy: start a new page, and reset the page number.  This way,
% the second copy of the abstract is not counted as separate pages.
% Uncomment the next 6 lines if you need two copies of the abstract
% page.
% \setcounter{page}{\thesavepage}
% \begin{abstractpage}
% % $Log: abstract.tex,v $
% Revision 1.1  93/05/14  14:56:25  starflt
% Initial revision
% 
% Revision 1.1  90/05/04  10:41:01  lwvanels
% Initial revision
% 
%
%% The text of your abstract and nothing else (other than comments) goes here.
%% It will be single-spaced and the rest of the text that is supposed to go on
%% the abstract page will be generated by the abstractpage environment.  This
%% file should be \input (not \include 'd) from cover.tex.

Let $G=(V,E)$ be an undirected unweighted graph on $n$ vertices and $m$ edges. We address the problem of sensitivity oracle for all-pairs mincuts in $G$ defined as follows.

Build a compact data structure that, on receiving a pair of vertices $s,t\in V$ and insertion/deletion of any edge as query, can efficiently report the value of the mincut between $s$ and $t$ upon the update.

To the best of our knowledge, there exists no data structure for this problem which takes $o(mn)$ space and a non-trivial query time. Recently, Baswana, Gupta, and Knollman \cite{DBLP:conf/esa/BaswanaGK20} gave a data structure that can handle single edge insertion in ${\cal O}(n^2)$ space and ${\cal O}(1)$ query time. We present the following results.

\begin{enumerate}
    \item We present a sensitivity oracle for all-pairs mincuts. Our data structure guarantees ${\cal O}(1)$ query time. The space occupied by this data structure is ${\cal O}(n^2)$ which matches the worst-case size of a graph on $n$ vertices. A resulting $(s,t)$-mincut after edge insertion/deletion can be reported in ${\cal O}(n)$ time which is optimal. Our data structure also subsumes the results of Baswana, Gupta, and Knollman \cite{DBLP:conf/esa/BaswanaGK20}.
    \item We give conditional lower bounds on data structures that can handle dual-deletions (or dual-faults) for a $(s,t)$-mincut. We also give a conditional lower bound on a data structure storing all static $(\{s,u\},\{t,v\})$-mincut values with fixed $s,t \in V$. This implies a conditional lower bound for a generalized flow tree for $2 \times 2$ mincuts, i.e. a data structure that can report the value of static $(\{s,u\},\{t,v\})$-mincut for given vertices $s,t,u,v \in V$.
\end{enumerate}

Some parts of this work have been taken from a recent research of Baswana and Pandey \cite{DBLP:journals/corr/BaswanaP20}, but presented here for sake of continuity. 

% \end{abstractpage}

\section*{Acknowledgments}

I am deeply indebted to Prof Surender Baswana for allowing me to work with him. I am thankful to him for having regular meetings despite his busy schedule and helping me acquire the necessary perseverance for solving a research problem. I would like to thank Prof Yefim Dinitz and Prof Alek Vainshtein for their seminal work ``The connectivity carcass of a vertex subset in a graph and its incremental maintenance" which appeared in STOC 1994 and subsequently in SIAM Journal of Computing 2000 edition. The paper truly captures the entire anatomy of mincuts and also forms the foundation of our results. Last but not the least, I would like to thank my parents and sister for their incredible and unconditional support in this pandemic. 
%%%%%%%%%%%%%%%%%%%%%%%%%%%%%%%%%%%%%%%%%%%%%%%%%%%%%%%%%%%%%%%%%%%%%%
% -*-latex-*-
