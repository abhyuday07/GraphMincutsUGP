\chapter{${\cal O}(n^2)$ space sensitivity oracle for all-pairs mincuts}


% {
% \color{blue}

% \begin{itemize}
% \item Edge-containment query for fixed s,t.
% \item Mincut containing $(x,y)$ using $(s,t)$-mincut for fixed $s,t$.
% \begin{itemize}
%     \item $O(m)$ space $O(m)$ time.
%     \item Augmented topological numbers.
%     \item $O(n)$ space $O(n)$ time.
% \end{itemize}

% \item Edge-containment query for $s,t \in S$ and $c_{s,t}=c_S$.
% \item Generalize $O(n)$ space $O(1)$ time.
% \item Mincut containing $(x,y)$ using $(s,t)$-mincut for $s,t \in S$ and $c_{s,t} = c_S$.
% \begin{itemize}
%     \item Idea of augmentation.
%     \item Given a stretched unit u and a bunch. Report the set of all stretched units that precede them in some topological ordering.
%     \item $O(n)$ space $O(n)$ time.
% \end{itemize}
% \end{itemize}
% }

% \boxed{
% $\tau$, $P$, $\pi$
% }

\section{Edge-containment query for fixed $s,t \in V$} \label{subsec:fixed-s-t}


Consider the problem of identifying if a given edge $(x,y)$ lies in a $(s,t)$-mincut for a designated pair of vertices $s,t \in V$. It is quite evident that an edge $(x,y)$ lies in a $(s,t)$-mincut if $x$ and $y$ are mapped to different nodes in the strip ${\cal D}_{s,t}$. This query can be reported in ${\cal O}(1)$ time by storing the node mapping of each vertex in ${\cal O}(n)$ space.

Reporting a $(s,t)$-mincut that contains edge $(x,y)$ requires more insights. Without loss in generality, assume that edge $(x,y)$ lies in side-$\mathbf t$ of $\mathbf x$. If $\mathbf x$ is the same as $\mathbf s$, the set of vertices mapped to node $\mathbf s$ define a $(s,t)$-mincut that contains $(x,y)$. Thus, assume that $\mathbf x$ is a non-terminal in the strip ${\cal D}_{s,t}$. It is important to observe that set of vertices mapped to nodes in the reachability cone of $\mathbf x$ towards $\mathbf s$, ${\cal R}_s({\mathbf x})$, defines a $(s,t)$-mincut that contains edge $(x,y)$. However, reporting this mincut requires ${\cal O}(m)$ time and ${\cal O}(m)$ space. 
% Assuming Conjecture \ref{conj:directed-reachability-hypothesis} holds, any improvement in query time would require ${\Omega}(n^2)$ space.

To achieve better space and query time, we report another $(s,t)$-mincut that contains $(x,y)$ and has a simpler structure. Suppose $\tau$ is a topological ordering on the node set of strip ${\cal D}_{s,t}$ with $\tau(\mathbf s) = 0$. We show that storing the node mapping of each vertex and topological number of each node $\tau$ of the strip ${\cal D}_{s,t}$ can be used to report a $(s,t)$-mincut efficiently. 
% Consider the set of nodes, $X = \{u \;|\; \tau(u) \leq \tau(\mathbf x)\}$, with topological numbers less than or equal to that of $\mathbf x$. The set of vertices mapped to nodes in $X$ defines one such $(s,t)$-mincut. 
This data structure takes only ${\cal O}(n)$ space and can report a $(s,t)$-mincut containing edge $(x,y)$ in ${\cal O}(n)$ time. We state the following lemma.

\begin{lemma}
\label{lem:s-t-mincut-containing-x-y-topological-fixed}
Consider the strip ${\cal D}_{s,t}$ with ${\mathbf x}$ as a non-terminal, and edge $(x,y)$ lies on side-${\mathbf t}$ of ${\mathbf x}$. Suppose $\tau$ is a topological order on the nodes in the strip. The set of vertices mapped to nodes in set $X = \{u \;|\; \tau(u) \leq \tau(\mathbf x)\}$ defines a $(s,t)$-mincut that contains edge $(x,y)$.
\end{lemma}

\section{Edge-containment query for Steiner set $S$} \label{subsec:edge-containment-ds-steiner}

Suppose $S$ is a designated Steiner set and $s,t\in S$ are Steiner vertices separated by some Steiner mincut. We can determine if an edge $(x,y)\in E$ belongs to some $(s,t)$-mincut using the strip ${\cal D}_{s,t}$ that can be built from the connectivity carcass. However, the construction of strip requires ${\cal O}(\min(m,nc_S))$ time. Interestingly, we show that only the skeleton and the projection mapping of the connectivity carcass are sufficient for answering this query in constant time. Moreover, the skeleton and the projection mapping occupy only ${\cal O}(n)$ space compared to the ${\cal O}(\min(m,nc_S))$ space occupied by the entire connectivity carcass.

Similar to the projection mapping of the stretched units, Dinitz and Vainshtein \cite{DBLP:journals/siamcomp/DinitzV00} introduced the notion of projection mapping for edges as follows. Suppose $(x,y)\in E$. If $x$ and $y$ belong to the same unit, then $P(x,y) = \varnothing$. If $x$ and $y$ belong to distinct terminal units mapped to nodes, say $\nu_1$ and $\nu_2$, in the skeleton ${\cal H}_S$, then $P(x,y) = P(\nu_1,\nu_2)$. If at least one of them belongs to a stretched unit, $P(x,y)$ is the extended path defined in Lemma \ref{lem:path-extendable}. This allows us to state the following lemma which follows from the construction of a strip corresponding to a subbunch given in Section \ref{subsec:connectivity-carcass}.

\begin{lemma}[\cite{DBLP:conf/stoc/DinitzV94}]
Edge $(x,y)\in E$ appears in the strip corresponding to a subbunch if and only if one of the structural edge in the cut of ${\cal H}_S$ corresponding to this subbunch lies in $P(x,y)$.
\label{lem:edge-path-intersect-subbunch}
\end{lemma}

We state the necessary and sufficient condition for an edge $(x,y)$ to lie in an $(s,t)$-mincut. Note that two paths are said to intersect in the skeleton if the unique path of cycle and tree edges in both the paths intersect at some cycle or tree edge.
% The skeleton ${\cal H}_{S(\nu)}$ and the corresponding projection mapping $\pi_{S(\nu)}$ have the sufficient information to infer whether any edge $(x,y)\in E$ belongs to some $(s,t)$-mincut as stated by the following lemma. We say that two paths intersect if the unique path of cycle and tree edges in both the paths intersect at some cycle or tree edge.

\begin{lemma} \label{lem:path-intersects-tree}
 Edge $(x,y)\in E$ belongs to a $(s,t)$-mincut if and only if the proper path $P(x,y)$ intersects a path between the nodes containing $s$ and $t$ in in $\mathcal H_{S}$.
\end{lemma}
\begin{proof}

% {\color{blue} make it short.}

% Observe that an edge $(x,y)$ lies in a $(s,t)$-mincut if and only if it appears in the strip ${\cal D}_{s,t}$ (follows from Lemma \ref{lem:E_y-edges-same-side}). Infact, we can extend this notion for subbunch as well. The edge $(x,y)$ lies in some $(s,t)$-mincut if and only if it appears in the strip corresponding to some subbunch that separates $s$ from $t$.

% Consider each subbunch that separates $s$ from $t$.

Let $\nu_1$ and $\nu_2$ be the nodes in ${\cal H}_S$ containing $s$ and $t$ respectively. A cut in ${\cal H}_S$ corresponding to any tree-edge (or pair of cycle edges in same cycle) in the path from $\nu_1$ to $\nu_2$ defines a subbunch separating $s$ from $t$. Moreover, it follows from the structure of the skeleton that no other cut in the skeleton corresponds to a subbunch separating $s$ from $t$. Suppose $(x,y)$ lies in some $(s,t)$-mincut. Thus, it must be in some subbunch separating $s$ from $t$. From the above discussion, we know that this subbunch must correspond to a cut in the path from $\nu_1$ to $\nu_2$ in skeleton ${\cal H}_S$. Moreover, it follows from Lemma \ref{lem:edge-path-intersect-subbunch} that $P(x,y)$ contains one of the structural edge in this cut. This implies that $P(x,y)$ intersects the path from $\nu_1$ to $\nu_2$ in skeleton ${\cal H}_S$.

Now, consider the other direction of this proof. Suppose $P(x,y)$ and the path from $\nu_1$ to $\nu_2$ intersect at some cycle (or tree edge) $c$. Let $e_1$ and $e_2$ be structural edges belonging to the cycle $c$ that are part of $P(x,y)$ and path from $\nu_1$ to $\nu_2$ respectively (in the case of tree edge $e_1=e_2=c$). Consider the cut in the skeleton corresponding to structural edges $e_1$ and $e_2$. It follows from Lemma \ref{lem:edge-path-intersect-subbunch} that $(x,y)$ lies in the strip corresponding to this subbunch. Since this cut separates $\nu_1$ from $\nu_2$ in ${\cal H}_S$, therefore the subbunch separates $s$ from $t$.
\end{proof}

We can check if paths $P(\nu_1,\nu_2)$ and $P(s,t)$ in the skeleton ${\cal H}_S$ intersect with ${\cal O}(1)$ LCA queries on skeleton tree ${\cal T}({\cal H}_S)$ (using Lemma \ref{lem:xyz}). Thus, we can store the skeleton tree ${\cal T}({\cal H}_S)$ and projection mapping in ${\cal O}(n)$ space and determine if an edge $(x,y)$ lies in an $(s,t)$-mincut for $c_{s,t}=c_S$ and $s,t \in S$ in ${\cal O}(1)$ time.


Reporting a $(s,t)$-mincut that contains edge $(x,y)$ again requires more insights. Assume that $P(s,t)$ and $P(\nu_1,\nu_2)$ intersect at some tree edge or cycle. Let $e$ be a tree or cycle-edge in proper path $P(\nu_1,\nu_2)$ that lies in intersection of these two paths. Suppose ${\cal B}$ is a subbunch corresponding to a cut in the skeleton ${\cal H}_S$ that contains $e$ and separates $s$ from $t$ and ${\cal D}_{\cal B}$ be the strip corresponding to this subbunch. Without loss in generality, assume that $\nu_1$ lies in the side of source $s$ in this strip (denoted by ${\mathbf s}$). Using Lemma \ref{lem:edge-path-intersect-subbunch}, it is evident that edge $(x,y)$ lies in strip ${\cal D}_{\cal B}$. Assume $\mathbf x$ is a stretched unit in this strip, otherwise the source $\mathbf s$ is the required $(s,t)$-mincut. Suppose $(x,y)$ lies in side-$\mathbf t$ of $\mathbf x$. The set of vertices mapped to ${\cal R}_s(\mathbf{x})$, i.e. reachability cone of $\mathbf x$ towards $\mathbf s$ in this strip, defines a $(s,t)$-mincut that contains edge $(x,y)$. However, reporting this mincut is a tedious task. We must have the flesh ${\cal F}_S$ to construct the strip ${\cal D}_{\cal B}$ and then report the set ${\cal R}_s(\mathbf{x})$. This would require ${\cal O}(m)$ space and ${\cal O}(m)$ time.

It is important to observe that the difficulty we face here is similar to the one highlighted in Section \ref{subsec:fixed-s-t}. To achieve better space and query time, we use similar ideas as used in Section \ref{subsec:fixed-s-t}. We aim to report another $(s,t)$-mincut that contains $(x,y)$ and has a simpler structure. In particular, we aim to report a set of units $X = \{ u \;|\; \tau_{\cal B}(u)\leq \tau_{\cal B}(\mathbf{x})\}$ for some topological ordering $\tau_{\cal B}$ of nodes in strip ${\cal D}_{\cal B}$. Using Lemma \ref{lem:s-t-mincut-containing-x-y-topological-fixed}, we know that set of vertices mapped to nodes in set $X$ defines a $(s,t)$-mincut that contains edge $(x,y)$. Clearly, storing topological order for each stretched unit for each possible bunch in which it appear as a non-terminal is not an option. This is because doing so will require a lot of space. Thus, we try to devise an algorithm to efficiently identify set $X$ on without compromising ${\cal O}(n)$ size of the data structure.

Consider the data structure formed by the following components -- ~(i) the skeleton tree ${\cal T}({\cal H}_S)$, ~(ii) the projection mapping $\pi_S$ of all units and ~(iii) a mapping $\tau$ from each stretched unit to a number. For all stretched units mapped to path $P(\nu_1,\nu_2)$, $\tau$ assigns a topological order on these stretched units as they appear in the $(\nu_1,\nu_2)$-strip. We now show how this additional augmentation will help us report a $(s,t)$-mincut containing edge $(x,y)$ efficiently.


In order to make the ideas more simple, we describe a transitive relation between proper paths on skeleton called \textit{extendable in a direction}.


% {\color{red} extendable in direction $\nu_2$ looks a bit. Try to redefine only in terms of projection mapping paths.}
% \subsubsection{Extendable in a direction}
\begin{definition}[Extendable in a direction]
% Suppose $u$ and $v$ are two stretched units projected to proper paths $P(\nu_1,\nu_2)$ and $P(\nu_3,\nu_4)$ respectively. $v$ is said to be extendable in direction $\nu_2$ of $u$ if proper paths $P(\nu_1,\nu_2)$ and $P(\nu_3,\nu_4)$ are extendable to a proper path $P(\nu,\nu')$ with $P(\nu_1,\nu_2)$ as the initial part and $P(\nu_3,\nu_4)$ as the final part.
Consider two proper paths $P_1 = P(\nu_1,\nu_2)$ and $P_2 = P(\nu_3,\nu_4)$. $P_2$ is said to be extendable from $P_1$ in direction $\nu_2$ if proper paths $P_1$ and $P_2$ are extendable to a proper path $P(\nu,\nu')$ with $P_1$ as the initial part and $P_2$ as the final part.
\label{def:extendable}
\end{definition}

% It follows from Theorem \ref{lem:path-extendable} that if the stretched unit $v$ is reachable from $u$ in the direction $\nu_2$ through a coherent path, then $v$ is extendable in direction $\nu_2$ from $u$.
Moreover, verifying if $P(\nu_3,\nu_4)$ is extendable from $P(\nu_1,\nu_2)$ in direction $\nu_2$ can be done in ${\cal O}(1)$ LCA queries on the skeleton tree.


Suppose stretched unit $\mathbf x$ is mapped to path $P(\nu,\nu')$. Consider the set $X$ formed by stretched units appearing as non-terminals in strip ${\cal D}_{\cal B}$ for which one of the following holds true -- (i) the stretched unit (say $v$) is mapped to $P(\nu,\nu')$ and $\tau(v) \leq \tau(\mathbf x)$, and (ii) the stretched unit is not mapped to $P(\nu,\nu')$ but $\pi_S(v)$ is extendable from $P(\nu,\nu')$ in direction $\nu$. We state the following lemma.

\begin{lemma}
The vertices mapped to units in ${\mathbf s}\cup X$ define a $(s,t)$-mincut and contains all the edges in side-$\mathbf t$ of the inherent partition of $\mathbf x$.
\end{lemma}
\begin{proof}
Consider $u \in X$ to be a non-terminal unit in ${\cal D}_{\cal B}$. We shall show that ${\cal R}_{\mathbf s}(u)\setminus \mathbf{s} \subseteq X$, i.e. reachability cone of $u$ towards source $\mathbf{s}$ in the strip ${\cal D}_{\cal B}$ avoiding $\mathbf s$ is a subset of $X$. It follows from the construction that $u$ is either projected to $P(\nu,\nu')$ or $\pi_S(u)$ is extendable in direction $\nu$ from $P(\nu,\nu')$. Consider any unit $v$ in ${\cal R}_{\mathbf s}(u)\setminus \mathbf{s}$. Suppose $u$ and $v$ are both projected to $P(\nu,\nu')$. Since, $v$ is reachable from $u$ in direction $\nu$, it follows that $\tau(v) < \tau(u) < \tau(\mathbf{x})$. Thus, $v \in X$. Now, suppose $v$ is not projected to $P(\nu,\nu')$. In this case, $\pi_S(v)$ is extendable from $P(\nu,\nu')$ in direction $\nu$ (from Theorem \ref{lem:path-extendable}). It follows from the transitivity of Definition \ref{def:extendable} that $\pi_S(v)$ is extendable from $\pi_S(u)$ in direction $\nu$. Thus, $v \in X$. Therefore, ${\mathbf s}\cup X$ defines a $(s,t)$-mincut (from Lemma \ref{lem:mincut-transversal}).

Now, we prove the second part of the lemma. Consider any edge $(x,z)$ in the side-$\mathbf{t}$. If $z$ is in $\mathbf t$ then $z \not \in X$ from the construction. Assume that $z$ is a non-terminal unit in ${\cal D}_{\cal B}$. If $z$ is projected to path $P(\nu,\nu')$ then $\tau(z) > \tau(u)$. Thus, $z \not \in X$. Otherwise $\pi_S(z)$ is extendable from $P(\nu,\nu')$ in direction $\nu'$. It follows from Definition \ref{def:extendable} that $z \not \in X$. Thus, the cut defined by ${\mathbf s} \cup X$ contains all edges in side-$\mathbf{t}$ of the inherent partition of $\mathbf x$.

\end{proof}

This data structure occupies ${\cal O}(n)$ space and can report a $(s,t)$-mincut containing edge $(x,y)$ in ${\cal O}(n)$ time. We shall use this data structure to build a sensitivity oracle for all-pairs mincuts.


\section{Edge-containment query for all-pairs Mincuts}

The hierarchical tree structure of Katz, Katz, Korman and Peleg \cite{DBLP:journals/siamcomp/KatzKKP04} can be suitably augmented to design a sensitivity oracle for all-pairs mincuts. We augment each internal node $\nu$ of the hierarchy tree ${\cal T}_G$ by the data structure discussed in Section \ref{subsec:edge-containment-ds-steiner} for the Steiner set $S(\nu)$. Determining if given edge $(x,y)$ lies in some $(s,t)$-mincut for given pair of vertices $s,t\in V$ can be done using Algorithm \ref{algo:quadratic-space-query} in constant time. 

\begin{algorithm}%[H]
    \caption{Single edge-containment queries in ${\cal O}(n^2)$ data structure}
    \label{algo:quadratic-space-query}
    \begin{algorithmic}[1] % The number tells where the line numbering should start
        \Procedure{edge-cotained}{$s,t,x,y$}
            \State{${\mu}\gets$ LCA($\mathcal T_G,s,t$)}
            \State $\mathcal P_1 \gets P(\pi_{S(\mu)}(s),\pi_{S(\mu)}(t))$
            \State $\mathcal P_2 \gets P(x,y)$
            \If{$\mathcal P_1 \cap \mathcal P_2 = \varnothing$} \Comment{Check if paths intersect} 
            \State \textbf{return} False
            \Else 
            \State \textbf{ return} True
            \EndIf
        \EndProcedure
    \end{algorithmic}
\end{algorithm}


% We state the following theorem.

% We build our data structure using the findings of Lemma \ref{lem:path-intersects-tree}. We augment each internal node $\mu$ of the hierarchy tree ${\cal T}_G$ with the skeleton tree $T({\cal H}_{S(\mu)})$ and the projection mapping ${\pi}_{S(\mu)}$ corresponding to Steiner set $S(\mu)$. In addition to it, for each stretched unit $u$ mapped to path $P(\nu_1,\nu_2)$ in skeleton ${\cal H}_{S(\mu)}$ we also store a number ${\tau(u)}$ that denotes the topological order of $u$ as it appears in the $(\nu_1,\nu_2)$-strip.  This additional information will help us efficiently report the mincut upon failure of an edge. Since augmentation at each internal node takes ${\cal O}(n)$ space, therefore the total space occupied by the data structure is only ${\cal O}(n^2)$.


% Consider any node $\nu$ in ${\cal T}_G$.
% Let $s,t$ be any two vertices such that
% $\nu$ is their LCA in ${\cal T}_G$. $s$ and $t$ must belong to different nodes in the the skeleton ${\cal H}_{S(\nu)}$ stored at $\nu$. 
% It follows from Lemma \ref{lem:path-intersects-tree} that for a single-edge-containment query, we just have to keep the skeleton tree and the projection mapping at each internal node in ${\cal T}_G$. It is important to note that both these data structures collectively only take up ${\cal O}(n)$ space, contrary to the ${\cal O}(\min(m,nc_S)$ size taken up by the entire connectivity carcass. Thus, the size taken up by complete data structure is only ${\cal O}(n^2)$. 

% \section{Edge-containment Query on Data Structure}
% Determining whether a given edge belongs to a $(s,t)$-mincut can be done as follows. Let $\mu$ be the LCA of $s$ and $t$ in ${\cal T}_G$. It follows from Observation \ref{obs:(s,t)-mincut-lca} that $c_{s,t}=c_{S(\mu)}$. Thus, $s$ and $t$ must be separated by some Steiner mincut for set $S(\mu)$. We check if paths $P(x,y)$ and $P(\pi_{S(\mu)}(s),\pi_{S(\mu)}(t))$ intersect in the skeleton ${\cal H}_{S(\mu)}$ (using Lemma \ref{lem:path-intersects-tree}). This can be done using ${\cal O}(1)$ LCA queries on the skeleton tree $T({\cal H}_{S(\mu)})$. Since it takes ${\cal O}(1)$ time for answering one LCA query \cite{DBLP:journals/jal/BenderFPSS05}, so the query time will be only ${\cal O}(1)$. Algorithm \ref{algo:quadratic-space-query} presents a concise pseudocode of the query answering algorithm.




% Now, we show how to report an $(s,t)$-mincut which contains the edge $(x,y)$ in ${\cal O}(n)$ time. 

% In order to make the ideas more simple, we describe a transitive relation between stretched units called \textit{extendable in a direction}.

% \begin{definition}[Extendable in a direction]
% Suppose $u$ and $v$ are two stretched units projected to proper paths $P(\nu_1,\nu_2)$ and $P(\nu_3,\nu_4)$ respectively. $v$ is said to be extendable in direction $\nu_2$ of $u$ if proper paths $P(\nu_1,\nu_2)$ and $P(\nu_3,\nu_4)$ are extendable to a proper path $P(\nu,\nu')$ with $P(\nu_1,\nu_2)$ as the initial part and $P(\nu_3,\nu_4)$ as the final part.
% % \label{def:extendable}
% \end{definition}

% It follows from Theorem \ref{lem:path-extendable} that if the stretched unit $v$ is reachable from $u$ in the direction $\nu_2$ through a coherent path, then $v$ is extendable in direction $\nu_2$ from $u$. Moreover, verifying if $v$ is extendable from $u$ in direction $\nu_2$ can be done in ${\cal O}(1)$ LCA queries on the skeleton tree.

% Suppose $s,t \in S$ are steiner vertices such that $c_{s,t}=c_S$ and $u$ is a stretched unit mapped to path $P(\nu_1,\nu_2)$. Moreover, it is known that $P(s,t)$ and $P(\nu_1,\nu_2)$ intersect at some tree edge or cycle. Let $e$ be a tree or cycle-edge in proper path $P(\nu_1,\nu_2)$ that. Our task is to report a $(s,t)$-mincut that contains all the edges in one side of the inherent partition of $u$ (say side-$\nu_2$). Suppose ${\cal B}$ is a subbunch corresponding to a cut in the skeleton ${\cal H}_S$ that contains $e$ and separates $s$ from $t$. Without loss in generality, assume that $\nu_1$ lies in the side of $s$ in this cut (denoted by ${\mathbf s}$). It follows from the construction that $u$ is a non-terminal unit in the strip ${\cal D}_{\cal B}$. Assume that side of $s$ is the source in this strip. 

% % To report a $(s,t)$-mincut containing all edges in side-$\nu_2$ of $u$, it suffices to report all non-terminal units in strip ${\cal D}_{\cal B}$ that precede $u$ in the topological ordering. It is important to note that multiple topological ordering may exist for this strip, but any one of them shall work in this case.

% Consider the set $U$ formed by stretched units appearing as non-terminals in strip ${\cal D}_{\cal B}$ for which at one of the following holds true -- (i) the stretched unit (say $v$) is mapped to $P(\nu_1,\nu_2)$ and $\tau(v) \leq \tau(u)$, and (ii) the stretched unit is not mapped to $P(\nu_1,\nu_2)$ but is extendable from $u$ in direction $\nu_1$. We state the following lemma.

% \begin{lemma}
% The vertices mapped to units in ${\mathbf s}\cup U$ define a $(s,t)$-mincut that contains all the edges in side-${\nu}_2$ of the inherent partition of $u$.
% \end{lemma}
% \begin{proof}
% Consider $x \in U$ to be a unit. We shall show that ${\cal R}_{\mathbf s}(x)\setminus \mathbf{s} \subseteq U$. It follows from the construction that $x$ is either projected to $P(\nu_1,\nu_2)$ or is extendable in direction $\nu_1$ from $x$. Consider any unit $y$ in ${\cal R}_{\mathbf s}(x)\setminus \mathbf{s}$. Suppose $x$ and $y$ are both projected to $P(\nu_1,\nu_2)$. Since, $y$ is reachable from $x$ in direction $\nu_1$, it follows that $\tau(y) < \tau(x) < \tau(u)$. Thus, $y \in U$. Now, suppose $y$ is not projected to $P(\nu_1,\nu_2)$. In this case, $y$ is extendable from $x$ in direction $\nu_1$ (from Theorem \ref{lem:path-extendable}). It follows from the transitivity of Definition \ref{def:extendable} that $y$ is extendable from $u$ in direction $\nu_1$. Thus, $y \in U$.

% Now, we prove the second part of the lemma. Consider any edge $(u,w)$ in the side-$\nu_2$. If $w$ is in $\mathbf t$ then $w \not \in U$ from the construction. Assume that $w$ is a non-terminal unit in ${\cal D}_{\cal B}$. If $w$ is projected to path $P(\nu_1,\nu_2)$ then $\tau(w) > \tau(u)$. Thus, $w \not \in U$. Otherwise $w$ is extendable from $u$ in direction $\nu_2$. It follows from Definition \ref{def:extendable} that $w \not \in U$. Thus, the cut defined by ${\mathbf s} \cup U$ contains all edges in side-$\nu_2$ of the inherent partition of $u$.

% \end{proof}

% Now, consider the case when at least one of $x$ and $y$ belongs to a stretched unit. In this case, we can use the above ideas to report the $(s,t)$-mincut containing this edge. If both $x$ and $y$ belong to different terminal units, then simply reporting the terminal unit $\mathbf{s}$ of ${\cal D}_{\cal B}$ will suffice.


\section{Edge-insertion query on Data Structure}

Determining whether insertion of an edge $(x,y)$ increases the $(s,t)$-mincut is comparatively simpler. Again, let $\mu$ be the LCA of $s$ and $t$ in ${\cal T}_G$. For sake of simplicity, assume that $x,y,s$ and $t$  be units in flesh graphs corresponding to vertices $x,y,s$ and $t$ respectively. Using Lemma \ref{lem:edge-insertion-increases-mincut}, we only need to determine if $x \in s_t^N$ and $y \in t_s^N$ or vice-versa. Using Lemma \ref{lem:u-nearest-s-t-mincut}, we can perform this operation in ${\cal O}(1)$ time. Reporting $(s,t)$-mincut after insertion of an edge can be done in ${\cal O}(n)$ time (using Lemma \ref{lem:u-nearest-s-t-mincut}) as $s_t^N$ is one such mincut. Thus, we can summarize the results of previous two sections in the following theorem.

\begin{theorem}
Given an undirected unweighted multigraph $G=(V,E)$ on $n=|V|$ vertices, there exists an
${\cal O}(n^2)$ size sensitivity oracle that can report the value of $(s,t)$-mincut for any $s,t \in V$ upon insertion/deletion of an edge. A $(s,t)$-mincut incorporating the insertion/deletion can be reported in ${\cal O}(n)$ time.
\label{thm:O(n^2)-size-data-structure}
\end{theorem}