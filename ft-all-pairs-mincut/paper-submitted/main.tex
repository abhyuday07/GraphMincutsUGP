
\documentclass[letterpaper,11pt]{article}
\usepackage[margin=1in]{geometry}
\usepackage{my_style}
\usepackage{subfiles}
% \usepackage{authblk}
\usepackage{float}
\usepackage{wrapfig}
\usepackage{microtype} %if unwanted, comment out or use option "draft"
% \usepackage{style/lipics-addon}

%\graphicspath{{./style/lipics/}}%helpful if your graphic files are in another directory
\bibliographystyle{plainurl}
%\usepackage[backend=biber, sorting=ynt]{biblatex}
%\addbibresource{refs.bib}
%\usepackage{csquotes}

% Tools for theorems and restates
\usepackage{thmtools}
\usepackage{thm-restate}

\usepackage{floatflt}
\usepackage{graphics}


% \bibliographystyle{alpha}
\begin{document}
\pagenumbering{gobble}
\begin{titlepage}

\title{Fault-Tolerant All-Pairs Mincuts}
\author{
  Surender Baswana\thanks{Department of Computer Science \& Engineering, IIT Kanpur, Kanpur -- 208016, India, sbaswana@cse.iitk.ac.in}
  \and
  Abhyuday Pandey\thanks{Department of Computer Science \& Engineering, IIT Kanpur, Kanpur -- 208016, India, abhyuday@cse.iitk.ac.in}
}
\maketitle

\begin{abstract}
Let $G=(V,E)$ be an undirected unweighted graph on $n$ vertices and $m$ edges. We address the problem of fault-tolerant data structure for all-pairs mincuts in $G$ defined as follows.

Build a compact data structure that, on receiving a pair of vertices $s,t\in V$ and any edge $(x,y)$ as query, can efficiently report the value of the mincut between $s$ and $t$ upon failure of the edge $(x,y)$.

To the best of our knowledge, there exists no data structure for this problem which takes $o(mn)$ space and a non-trivial query time. We present two compact data structures for this problem.
\begin{enumerate}
\item Our first data structure guarantees ${\cal O}(1)$ query time. The space occupied by this data structure is ${\cal O}(n^2)$ which matches the worst-case size of a graph on $n$ vertices.
\item
Our second data structure takes ${\cal O}(m)$ space which 
% is optimal
matches the size of the graph. The query time is ${\cal O}(\min(m,n c_{s,t}))$ where $c_{s,t}$ is the value of the mincut between $s$ and $t$ in $G$. The query time guaranteed by our data structure is faster by a factor of $\Omega(\sqrt{n})$ compared to the best known algorithm \cite{DBLP:conf/focs/GoldbergR97a,DBLP:conf/stoc/KargerL98} to compute a $(s,t)$-mincut.
\end{enumerate}


Both these data structures can also report the resulting $(s,t)$-mincut incorporating the failure in ${\cal O}(\min(m,n c_{s,t}))$ time.

\end{abstract}
\end{titlepage}
\pagebreak
\pagenumbering{arabic}
\section{Introduction}
\subfile{src/introduction}

\section{Preliminaries} \label{sec:prelimiaries}
\subfile{src/preliminaries}

\section{Insights into \texorpdfstring{$3$}{3}-vertex mincuts} \label{sec:query-transformation}
\subfile{src/compact-graph-query-transf}


\vspace{-6mm}
\section{A Compact Graph for Query Transformation}
\vspace{-2mm}
\subfile{src/ft-steiner-connectivity}
\vspace{-3mm}
\section{Compact Data Structures for Edge-Containment Query} \label{sec:final-ds}
\vspace{-2mm}
\subfile{src/graph-contractions}


% \section{Conclusion and Future Work}

\pagebreak

\bibliography{refs}


% \pagebreak
% \section{TODO}
% \subfile{src/todo}

\pagebreak
\appendix

\subfile{src/appendix}

\end{document}