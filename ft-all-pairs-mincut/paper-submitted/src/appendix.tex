% \section{Proof of Lemma \ref{lem:lower-bound}}
% \label{appendix:lower-bound}
% \begin{proof}
% Let $N = {n \choose 2}$ be the maximum possible number of edges in a graph. The number of undirected unweighted graph on $n$ vertices and $\frac{m}{3}$ edges is given by  ${N \choose \frac{m}{3}}$. Clearly, this number will be less than than the number of undirected unweighted graphs on \textit{atmost} $m$ edges. Now,
% \begin{equation}
%     \begin{split}
%         \nonumber
%         {N \choose \frac{m}{3}} &= \frac{N (N -1)...(N -(\frac{m}{3}-1))}{\frac{m}{3}!}\\
%         &\geq (3.\frac{N +1}{m} - 1)^{\frac{m}{3}}\\
%         &\geq 2^{\frac{m}{3}} \;\;\;\text{( Since $\frac{N +1}{m} > 1$)}
%     \end{split}
% \end{equation} 
% Since there are at least $2^{\frac{m}{3}}$ undirected unweighted graph on $n$ vertices and atmost $m$ edges, we will require $\Omega(m)$ space for storing at least one such graph. Hence, proved.
% \end{proof}


\section{Comparison of Query Time with Static Algorithms}
\label{appendix:non-trivial-query-time}
Goldberg and Rao \cite{DBLP:conf/focs/GoldbergR97a} gave a deterministic algorithm for computing $(s,t)$-mincut in an undirected unweighted graph that runs in ${\cal O}(n^{3/2}m^{1/2})$ time. Karger and Levine \cite{DBLP:conf/stoc/KargerL98} gave an algorithm that runs in ${\cal O}(nm^{2/3}c_{s,t}^{1/6})$ time for the same problem. 

We show that the query time of our data structure is ${\Omega}(\sqrt{n})$ times faster than both of them. The query time we achieve is ${\cal O}(\min(m,nc_{s,t}))$.
It is clear that $\frac{n^{3/2}m^{1/2}}{m}\geq \sqrt{n}$ and thus, the query time is $\Omega(\sqrt{n})$ times faster than Goldberg and Rao's algorithm \cite{DBLP:conf/focs/GoldbergR97a}. We can show the same for Karger and Levine's algorithm \cite{DBLP:conf/stoc/KargerL98} by separately considering the cases $m<nc_{s,t}$ and $m\geq nc_{s,t}$.

\section{Proof of Lemma \ref{lem:3-vertex-lemma} (3-vertex lemma)} \label{appendix:3-vertex-lemma}

\begin{proof}
Let $\alpha = c(\bar{A}\cap B, A\cap {B}),
~\beta = c(\bar{A} \cap B, A\cap \bar{B}),~\gamma=c(\bar{A} \cap B, \bar{A}\cap \bar{B})$.
Refer to Figure \ref{fig:non-S-crossing} (i) that illustrates these edges and the respective cuts.
\begin{figure}[H]
\centering
\includegraphics[width=0.75\textwidth]{src/images/S-crossing-and-non-crossing_new.pdf}
    \caption{(i) $\alpha$, $\beta$ and $\gamma$ denote the capacities of edges incident on $\bar{A}\cap B$ from $A\cap B$, $A\cap \bar{B}$, and $\bar{A} \cap \bar{B}$ respectively. (ii) There are no edges along the diagonal between $\bar{A}\cap {B}$ and $A\cap \bar{B}$.}
\label{fig:non-S-crossing}
\end{figure}

Applying Lemma \ref{lem:subset-property-of-min-cut}
on $(s,t)$-mincut with $S=A$ and $S'= \bar{A}\cap B$, we get
\begin{equation}
    \alpha + \beta \le \gamma 
\label{eq:alpha+beta-le-gamma}
\end{equation}
Applying Lemma \ref{lem:subset-property-of-min-cut} on $(r,s)$-mincut with $S=\bar B$ and  $S'= \bar{A} \cap B$, we get $\gamma + \beta \le \alpha$. This inequality combined with Inequality \ref{eq:alpha+beta-le-gamma} implies that $\beta=0$. That is, $c(\bar{A}\cap B, A\cap \bar{B})=0$. This completes the proof of Assertion (1). Refer to Figure \ref{fig:non-S-crossing} (ii) for an illustration. 
It follows from (1) that $\alpha=\gamma$.
That is, $\bar{A} \cap B$ has equal number edges incident from $\bar{A}\cap \bar{B}$ as from $A\cap B$.  This fact can be easily used to infer Assertions (2) and (3) by 
removing $\bar{A}\cap B$ from $B$ and
$\bar{A}$ respectively.
\end{proof}

\section{Compact representation for all global mincuts} \label{appendix:cactus}

Let $c_V$ denote the value of the global mincut of the graph $G$.
Dinitz, Karzanov, and Lomonosov \cite{DL76} showed that there exists a graph ${\cal H}_V$ of size $O(n)$ that compactly stores all global mincuts of $G$. 
%In order to maintain the distinction between the two graphs,
Henceforth, we shall use nodes and structural edges for vertices and edges of ${\cal H}_V$ respectively. There exists a projection mapping $\pi:V(G)\rightarrow V({\cal H}_V)$ assigning a vertex of graph $G$ to a node in graph ${\cal H}_V$. In this way, any cut $(A,{\bar A})$ in cactus ${\cal H}_V$ is associated to a cut $(\pi^{-1}(A),\pi^{-1}(\bar A))$ in the original graph $G$.
The graph ${\cal H}_V$ has a nice tree-like structure with the following properties.
\begin{enumerate}
    \item Any two distinct simple cycle of ${\cal H}_V$ have at most a node in common. This is equivalent to the property that each structural edge of ${\cal H}_V$ belongs to at most one simple cycle. Each cut in ${\cal H}_V$ either corresponds to a tree edge or a pair of cycle edges in the same cycle.
    \item If a stuctural edge belongs to a simple cycle, it is called a \textit{cycle edge} and its weight is $\frac{c_V}{2}$. Otherwise, the structural edge is called a \textit{tree edge} and its weight is $c_V$.
    \item For any cut in the cactus ${\cal H}_V$, the associated cut in graph $G$ is a global mincut. Moreover, any global mincut in $G$ must have at least one associated cut in ${\cal H}_V$.
\end{enumerate}

Let $\nu$ and $\mu$ be any two nodes in the cactus ${\cal H}_V$. If they belong to the same cycle, say $c$, there are two paths between them on the cycle $c$ itself - their union forms the cycle itself. Using the fact that any two cycles in  ${\cal H}_V$ can have at most one common node, it can be seen that these are the only paths between $\nu$ and $\mu$. Using the same fact, if $\nu$ and $\mu$ are two arbitrary nodes in the cactus, there exists a unique path of cycles and tree edges between these two nodes. Any global mincut that separates $\nu$ from $\mu$ must correspond to a cut in this path.

\subsection*{Construction of $(s,t)$-strip from cactus}
\label{sec:construction-strip-cactus}
Suppose $s,t \in V$ are two vertices such that $c_{s,t}$ is same as the global mincut value. 
So, each transversal of strip ${\cal D}_{s,t}$ corresponds to a global mincut that separates $s$ and $t$. Recall that cactus ${\cal H}_V$ stores all global mincuts. So we just need to contract it suitably so that only those cuts remain that separate $s$ and $t$. For this purpose,
we compute the path of cycles and tree edges between the nodes corresponding to $s$ and $t$ respectively. We compress each of the subcactus rooted to this path to a single vertex. The resultant graph we obtain will be the strip ${\cal D}_{s,t}$. The inherent partition of all the non-terminal units can be determined using the endpoints of the edges in the path.

\subsection*{Tree representation for cactus}

%  Let $G=(V,E)$ be an undirected graph, $S\subseteq V$ be the Steiner set of vertices and let ${\cal H}_S$ be the cactus graph storing all bunches of Steiner mincuts. 

We shall now show that ${\cal H}_V$ can be represented as a tree structure. This tree structure was also used by Dinitz and Westbrook in \cite{DBLP:journals/algorithmica/DinitzW98}. This representation will simplify our analysis on the cactus.

We now provide the details of the graph structure $T({\cal H}_V)$ that represents ${\cal H}_V$. The vertex set of $T({\cal H}_V)$ consists of all the cycles and the nodes of the cactus. For any node $\nu$ of the cactus ${\cal H}_V$, let $v(\nu)$ denote the corresponding vertex in $T({\cal H}_V)$. Likewise, for any cycle $\pi$ in the cactus, let $v(\pi)$ denote the corresponding vertex in $T({\cal H}_V)$. We now describe the edges of  $T({\cal H}_V)$. Let $\nu$ be any node of ${\cal H}_V$. Suppose there are $j$ cycles - $\pi_1,\ldots,\pi_j$ that pass through it. We add an edge between $v(\nu)$ and $v(\pi_i)$ for each $1\le i\le j$. Lastly, for each vertex $\nu(\pi)$ in $T{({\cal H}_V)}$ we store all its neighbours in the order in which they appear in the cycle $\pi$ in ${\cal H}_V$. This is done to ensure that information about the order of vertices in each cycle is retained. This complete the description of $T({\cal H}_V)$. For a better understanding, the reader may refer to Figure \ref{fig:transform-cactus-to-tree} that succinctly depicts the transformation carried out at a node $\nu$ of the cactus graph to build the corresponding graph structure $T({\cal H}_V)$. 

The fact that the graph structure $T({\cal H}_V)$ is a tree follows from the property that any two cycles in a cactus may have at most one vertex in common. Let us root $T({\cal H}_V)$ at any arbitrary vertex, say $v(\nu)$, for some node $\nu$ of ${\cal H}_V$. Since each cycle in ${\cal H}_V$ has at least 3 vertices, so each vertex corresponding to a cycle of ${\cal H}_V$ will have at least 2 children each corresponding to distinct nodes of ${\cal H}_V$. This also shows that the number of cycles in ${\cal H}_V$ is at most half of the number of nodes in ${\cal H}_V$. Hence, the size of $T({\cal H}_V)$ is of the order of the number of nodes of ${\cal H}_V$. 

\begin{figure}[H]
\centering
\includegraphics[width=0.6\textwidth]{src/images/Cactus-transformation.png}
    \caption{Transformation of cactus ${\cal H}_V$ to the tree $T({\cal H}_V)$.}
\label{fig:transform-cactus-to-tree}
\end{figure}

We know that if $\nu$ and $\mu$ are two nodes in the cactus, there exists a unique path of cycles and tree edges between them. It follows from the construction of $T({\cal H}_V)$ that the unique path between the vertices $v(\nu)$ and $v(\mu)$ captures the same path. Thus we state the following lemma.

\begin{lemma}
Let $\nu,\mu$ be any two arbitrary nodes in the cactus 
${\cal H}_V$. The unique path between $v(\nu)$ and $v(\mu)$ in $T({\cal H}_V)$ concisely captures all
paths between $\nu$ and $\mu$ in ${\cal H}_S$.
\label{lem:path-in-T(H_S)}
\end{lemma}

% Let $\nu$ and $\mu$ be any two nodes in skeleton ${\cal H}_S$. If they belong to the same cycle, say $c$, there are two paths between them on the cycle $c$ itself - their union forms the cycle itself. Using the fact that any two cycles in  ${\cal H}_S$ can have at most one common node, it can be seen that these are the only paths between $\nu$ and $\mu$. Using the same fact, if $\nu$ and $\mu$ belong to different cycles, there exists a unique sequence of alternating cycles and nodes $\langle \nu_1,c_1,\ldots,\nu_r,c_r,\nu_{r+1}\rangle $ satisfying the following 2 properties. \begin{itemize}
%     \item $\nu_1=\nu$, $\nu_{r+1}=\mu$, and for each $1< i\le r$, $\nu_i$ is the unique node common to $c_{i-1}$ and $c_i$.
%     \item Each path between $\nu$ and $\mu$ can be seen as a sequence $\langle p_1,\ldots p_r\rangle$ such that $p_i$ is a path between $\nu_i$ and $\nu_{i+1}$ on cycle $c_{i}$.
% \end{itemize}
% It follows from the construction of $T({\cal H}_S)$ that $\langle v(\nu_1),v(c_1),\ldots,v(\nu_r),v(c_r),v(\nu_{r+1})\rangle$ is the path between $v(\nu)$
% and $v(\mu)$. 
% Thus we can state the following lemma.
% \begin{lemma}
% Let $\nu,\mu$ be any two arbitrary nodes in the cactus 
% ${\cal H}_S$. The unique path between $v(\nu)$ and $v(\mu)$ in $T({\cal H}_S)$ concisely captures all
% paths between $\nu$ and $\mu$ in ${\cal H}_S$.
% \label{lem:path-in-T(H_S)}
% \end{lemma}

We root the tree $T({\cal H}_V)$ at any arbitrary vertex and augment it suitably so that it can answer any LCA query in $\mathcal O(1)$ time using \cite{DBLP:journals/jal/BenderFPSS05}. Henceforth, we shall use skeleton tree $T({\cal H}_S)$ to denote this data structure.

\section{Proof of Lemma \ref{lem:AUB-contains-E_y}} \label{appendix:AUB-contains-E_y}
\begin{proof}
It follows from Lemma \ref{lem:3-vertex-lemma}(3) that $A\cup B$ will be a $(s,t)$-mincut. Hence $A\cup B$ will be a transversal in strip ${\cal D}_{A,t}$ that stores all
$(s,t)$-mincuts.
From definition, $y$ belongs to $\bar{A}$. Refer to Figure \ref{fig:non-S-crossing}($ii$).  If $y\in \bar{A}\cap B$, then it follows from Lemma \ref{lem:3-vertex-lemma}(1) that all neighbors of $y$ corresponding to $E_y$ will belong to $\bar{A}\cap \bar{B}$. So $E_y$ belongs to the cut defined by $A\cup B$. The same holds for the case $y\in \bar{A}\cap\bar{B}$ as well since $B\subset A\cup B$.
\end{proof}


\section{Proof of Lemma \ref{lem:contracted-subcactus-mincut}} \label{appendix:contracted-subcactus-mincut}

Let $c$ be any cycle (or tree edge) passing through (incident on) $\nu$ in the skeleton
${\cal H}_S$. Let ${\cal D}_{s,t}$ be the strip corresponding to the sub-bunch defined by the structural edge(s) incident on $\nu$ by $c$.
Let $\nu$ be on the side of the source $\mathbf{s}$ in this strip.
Let ${\cal H}_S(c)$ be the subcactus formed by removing the structural edge(s) from $c$ incident on $\nu$ and not containing $\nu$. 
Recall that the subcactus ${\cal H}_S(c)$ was contracted into a vertex, say $v_c$, in the graph 
$G_{S'}$.
%Moreover, suppose $v_c$ is the contracted vertex corresponding to contracted subcactus ${\cal H}_S(c)$.
%Also, assume that $\nu$ is on the side of the source $\mathbf{s}$ in this sub-bunch.

\begin{lemma}
Let $u$ and $u'$ be any two non-terminal units in ${\cal D}_{s,t}$ such that none of them is compressed to $v_c$ in $G_{S'}$. If one of them is reachable from the other in the direction of ${\mathbf{s}}$, then both of them will be compressed to the same contracted vertex in $G_{S'}$.
%
%Let $u$ and $u'$ be any two non-terminal units in ${\cal D}_{s,t}$ such that $u'$ is reachable from $u$ in the direction of ${\mathbf{s}}$ and $u$ is not contracted to vertex $v_c$. $u'$ and $u$ will be compressed to the same contracted node in $G_{S'}$.
\label{lem:u-u'-in-G-nu}
\end{lemma}
\begin{proof}
Assume without loss of generality that $u'$ is reachable from $u$ in the direction of ${\mathbf{s}}$.
Let the proper paths associated with each of $u$ and $u'$ in ${\cal H}_S$ be $P(\nu_1,\nu_2)$ and $P(\nu_1',\nu_2')$ respectively. 
It follows from the construction of ${\cal D}_{s,t}$ that
$P(\nu_1,\nu_2)$ as well as $P(\nu_1',\nu_2')$ will pass through one of the structural edge(s) from $c$ on $\nu$. Without loss of generality,  assume that $P(\nu_1,\nu_2)$ passes through $e$. Since $P(\nu_1,\nu_2)$ is a proper path, this implies that this is the only structural edge in this cut (of skeleton) through which this path passes.
Since $u'$ is reachable from $u$ in flesh ${\cal F}_S$, so $P(\nu_1',\nu_2')$ will also have to pass through $e$ (from Lemma \ref{lem:path-extendable}).
It again follows from Lemma \ref{lem:path-extendable}, that $P(\nu_1,\nu_2)$ as well as $P(\nu_1',\nu_2')$ are subpaths of a path, say $P(\nu',\nu'')$, in skeleton
${\cal H}_S$. This combined with the above discussion establishes that $P(\nu',\nu'')$ has the structure shown in Figure \ref{fig:structure-of-p(nu',nu'')}.

Observe that any path in skeleton that passes through a node $\nu$ can intersect at most 2 cycles or tree-edges that are passing though $\nu$. We know that suffix of $P(\nu',\nu'')$ after $e$ lies in ${\cal H}_S(c)$, so the prefix upto $e$ must have endpoint in subcactus ${\cal H}_S(c')$ where $c'\neq c$. This implies that $u$ must be compressed to $v_{c'}$ because it is not compressed to $v_c$. Thus, $c'$ precedes $c$ in total order. It follows from the structure of path $P(\nu_1',\nu_2')$ that it will have an endpoint in ${\cal H}_S(c')$. Thus, $u'$ will be compressed to the same compressed vertex $v_{c'}$ in $G_{S'}$. This completes the proof.

\begin{figure}%[H]
\centering
\includegraphics[width=0.95\textwidth]{src/images/proof-reachability-compressed.pdf}
    \caption{The structure of path $P(\nu',\nu'')$.}
\label{fig:structure-of-p(nu',nu'')}
\end{figure}

\end{proof}
% Lemma \ref{lem:contracted-subcactus-mincut} implies that vertex set compressed to each contracted node defines a Steiner mincut. This completes the proof of Lemma \ref{lem:u-u'-in-G-nu}.

Consider the set of non-terminals in the strip ${\cal D}_{s,t}$ that are not compressed to contracted vertex $v_c$. Let this set be $U$. Observe that the set of units $\bigcup_{u\in U} {\cal R}_s(u)$ form a Steiner mincut (using Lemma \ref{lem:reachability-cones}). Moreover, it follows from Lemma \ref{lem:u-u'-in-G-nu} that each non-terminal unit in the set $\bigcup_{u\in U} {\cal R}_s(u)$ is not compressed to contracted vertex $v_c$. Thus, $U = \bigcup_{u\in U} {\cal R}_s(u) \setminus \{\mathbf{s}\}$. All the set of vertices compressed to $v_c$ forms the complement of set $\bigcup_{u\in U} {\cal R}_s(u)$, and thus defines the same Steiner mincut. Therefore, the set of vertices corresponding to each contracted vertex defines a Steiner mincut.

It follows from the construction that $G_{S'}$ is a quotient graph of $G$. Moreover, the number of contracted vertices equals the number of cycles and tree edges incident on node $\nu$ in the skeleton. Figure \ref{fig:image-contraction} gives a nice illustration of the contraction procedure.

\begin{figure}
    \centering
    \includegraphics[width=0.9\textwidth]{src/images/image_contraction.pdf}{}
    \caption{$2$-step contraction procedure to construct $G_{S'}$. We only show the vertices and relevant edges of graph along with the skeleton ${\cal H}_S$. Solid vertices belong to Steiner set $S$ and hollow vertices are non-Steiner vertices. All Steiner vertices inside node $\nu$ form the set $S'$.}
    \label{fig:image-contraction}
\end{figure}

\section{Proof of Lemma \ref{lem:linear-time-qt} and \ref{lem:mincut-qt}}
\label{appendix:linear-time-qt}
\begin{proof}
Consider the case when $y$ does not belong to any contracted vertex. In this case, all edges in $E_y$ remain intact in $G_{S'}$ and thus $E_{y'}=E_{y}$.

Now, suppose $y$ belong to contracted vertex $y'$. Let $\bar A$ be the set of vertices compressed to contracted vertex $y'$. We select a vertex $t\in {\bar A} \cap S$ and construct the ${\cal D}_{A,t}$ strip using the flesh ${\cal F}_S$ and skeleton ${\cal H}_S$ in time linear in the size of flesh (using Lemma \ref{lem:strip-from-carcass}). Using the construction outlined in Lemma \ref{lem:query-transformation} we can obtain the set of edges $E_{A}$ by computing reachability cone(s) in strip ${\cal D}_{A,t}$. This takes time linear in size of ${\cal D}_{A,t}$. All edges in $E_A$ share same endpoint $y'$ in $G_{S'}$. Thus, we get the set of edges $E_{y'}$ which is simply all edges in $G_{S'}$ corresponding to set $E_A$. Clearly, this process can be accomplished in time linear in the size of flesh ${\cal F}_S$.

Suppose we have a $(r,s)$-mincut in $G_{S'}$, say $B$ such that $s,t \not\in B$ that contains all edges in $E_{y'}$. If $y$ does not belong to any contracted vertex, this cut itself can be reported as $E_y=E_{y'}$. Suppose $y$ belong to contracted vertex $y'$. We can construct another $(r,s)$-mincut $B\cup R$ (recall the definition of $R$ in Proof of Lemma \ref{lem:query-transformation}). This procedure also involves construction of ${\cal D}_{A,t}$ strip and computation of reachability cone(s) in this strip. This process can also be accomplished in time linear in the size of flesh ${\cal F}_S$.
\end{proof}

\section{\texorpdfstring{An ${\cal O}(n^2)$}{A quadratic space} data structure for single edge-containment queries} \label{appendix:n2-ds}

% We build our ${\cal O}(n^2)$ data structure by augmenting each internal node $\nu$ of hierarchy tree ${\cal T}_G$ with the skeleton ${\cal H}_{S(\nu)}$ and projection mapping $\pi_{S(\nu)}$.

% Let $s,t$ be any two vertices such that $\nu$ is their LCA in ${\cal T}_G$. $s$ and $t$ must belong to different nodes in the the skeleton ${\cal H}_{S(\nu)}$ stored at $\nu$. Interestingly, the skeleton ${\cal H}_{S(\nu)}$ and the corresponding projection mapping $\pi_{S(\nu)}$ have sufficient information to infer whether any edge $(x,y)\in E$ belongs to some $(s,t)$-mincut. 
Suppose $S$ is a designated Steiner set and $s,t\in S$ are Steiner vertices separated by some Steiner mincut. We can determine if an edge $(x,y)\in E$ belongs to some $(s,t)$-mincut using the strip ${\cal D}_{s,t}$ that can be built from the connectivity carcass. However, the construction of strip requires ${\cal O}(\min(m,nc_S))$ time. Interestingly, we show that only the skeleton and the projection mapping of the connectivity carcass are sufficient for answering this query in constant time. Moreover, the skeleton and the projection mapping occupy only ${\cal O}(n)$ space compared to the ${\cal O}(\min(m,nc_S))$ space occupied by the entire connectivity carcass.


Similar to the projection mapping of the stretched units, Dinitz and Vainshtein \cite{DBLP:journals/siamcomp/DinitzV00} introduced the notion of projection mapping for edges as follows. Suppose $(x,y)\in E$. If $x$ and $y$ belong to the same unit, then $P(x,y) = \varnothing$. If $x$ and $y$ belong to distinct terminal units mapped to nodes, say $\nu_1$ and $\nu_2$, in the skeleton ${\cal H}_S$, then $P(x,y) = P(\nu_1,\nu_2)$. If at least one of them belongs to a stretched unit, $P(x,y)$ is the extended path defined in Lemma \ref{lem:path-extendable}. This allows us to state the following lemma which follows from the construction of a strip corresponding to a subbunch given in Section \ref{subsec:connectivity-carcass}.

\begin{lemma}[\cite{DBLP:conf/stoc/DinitzV94}]
Edge $(x,y)\in E$ appears in the strip corresponding to a subbunch if and only if one of the structural edge in the cut of ${\cal H}_S$ corresponding to this subbunch lies in $P(x,y)$.
\label{lem:edge-path-intersect-subbunch}
\end{lemma}

We state the necessary and sufficient condition for an edge $(x,y)$ to lie in an $(s,t)$-mincut. Note that two paths are said to intersect in the skeleton if the unique path of cycle and tree edges in both the paths intersect at some cycle or tree edge.
% The skeleton ${\cal H}_{S(\nu)}$ and the corresponding projection mapping $\pi_{S(\nu)}$ have the sufficient information to infer whether any edge $(x,y)\in E$ belongs to some $(s,t)$-mincut as stated by the following lemma. We say that two paths intersect if the unique path of cycle and tree edges in both the paths intersect at some cycle or tree edge.

\begin{lemma} \label{lem:path-intersects-tree}
 Edge $(x,y)\in E$ belongs to a $(s,t)$-mincut if and only if the proper path $P(x,y)$ intersects a path between the nodes containing $s$ and $t$ in in $\mathcal H_{S}$.
\end{lemma}
\begin{proof}
Observe that an edge $(x,y)$ lies in a $(s,t)$-mincut if and only if it appears in the strip ${\cal D}_{s,t}$ (follows from Lemma \ref{lem:E_y-edges-same-side}). Infact, we can extend this notion for subbunch as well. The edge $(x,y)$ lies in some $(s,t)$-mincut if and only if it appears in the strip corresponding to some subbunch that separates $s$ from $t$.

Consider each subbunch that separates $s$ from $t$. Let $\nu_1$ and $\nu_2$ be the nodes in ${\cal H}_S$ containing $s$ and $t$ respectively. A cut in ${\cal H}_S$ corresponding to any tree-edge (or pair of cycle edges in same cycle) in the path from $\nu_1$ to $\nu_2$ defines a subbunch separating $s$ from $t$. Moreover, it follows from the structure of the skeleton that no other cut in the skeleton corresponds to a subbunch separating $s$ from $t$. 

Suppose $(x,y)$ lies in some $(s,t)$-mincut. Thus, it must be in some subbunch separating $s$ from $t$. From the above discussion, we know that this subbunch must correspond to a cut in the path from $\nu_1$ to $\nu_2$ in skeleton ${\cal H}_S$. Moreover, it follows from Lemma \ref{lem:edge-path-intersect-subbunch} that $P(x,y)$ contains one of the structural edge in this cut. This implies that $P(x,y)$ intersects the path from $\nu_1$ to $\nu_2$ in skeleton ${\cal H}_S$.

Now, consider the other direction of this proof. Suppose $P(x,y)$ and the path from $\nu_1$ to $\nu_2$ intersect at some cycle (or tree edge) $c$. Let $e_1$ and $e_2$ be structural edges belonging to the cycle $c$ that are part of $P(x,y)$ and path from $\nu_1$ to $\nu_2$ respectively (in the case of tree edge $e_1=e_2=c$). Consider the cut in the skeleton corresponding to structural edges $e_1$ and $e_2$. It follows from Lemma \ref{lem:edge-path-intersect-subbunch} that $(x,y)$ lies in the strip corresponding to this subbunch. Since this cut separates $\nu_1$ from $\nu_2$ in ${\cal H}_S$, therefore the subbunch separates $s$ from $t$.
\end{proof}

We build our data structure using the findings of Lemma \ref{lem:path-intersects-tree}. We augment each internal node $\mu$ of the hierarchy tree ${\cal T}_G$ with the skeleton tree $T({\cal H}_{S(\mu)})$ and the projection mapping ${\pi}_{S(\mu)}$ corresponding to Steiner set $S(\mu)$. Since augmentation at each internal node takes ${\cal O}(n)$ space, therefore the total space occupied by the data structure is only ${\cal O}(n^2)$.


% Consider any node $\nu$ in ${\cal T}_G$.
% Let $s,t$ be any two vertices such that
% $\nu$ is their LCA in ${\cal T}_G$. $s$ and $t$ must belong to different nodes in the the skeleton ${\cal H}_{S(\nu)}$ stored at $\nu$. 
% It follows from Lemma \ref{lem:path-intersects-tree} that for a single-edge-containment query, we just have to keep the skeleton tree and the projection mapping at each internal node in ${\cal T}_G$. It is important to note that both these data structures collectively only take up ${\cal O}(n)$ space, contrary to the ${\cal O}(\min(m,nc_S)$ size taken up by the entire connectivity carcass. Thus, the size taken up by complete data structure is only ${\cal O}(n^2)$. 

Determining whether a given edge belongs to a $(s,t)$-mincut can be done as follows. Let $\mu$ be the LCA of $s$ and $t$ in ${\cal T}_G$. It follows from Observation \ref{obs:(s,t)-mincut-lca} that $c_{s,t}=c_{S(\mu)}$. Thus, $s$ and $t$ must be separated by some Steiner mincut for set $S(\mu)$. We check if paths $P(x,y)$ and $P(\pi_{S(\mu)}(s),\pi_{S(\mu)}(t))$ intersect in the skeleton ${\cal H}_{S(\mu)}$ (using Lemma \ref{lem:path-intersects-tree}). This can be done using ${\cal O}(1)$ LCA queries on the skeleton tree $T({\cal H}_{S(\mu)})$. Since it takes ${\cal O}(1)$ time for answering one LCA query \cite{DBLP:journals/jal/BenderFPSS05}, so the query time will be ${\cal O}(1)$ only. Algorithm \ref{algo:quadratic-space-query} presents a concise pseudocode of the query answering algorithm.

\begin{algorithm}%[H]
    \caption{Single edge-containment queries in ${\cal O}(n^2)$ data structure}
    \label{algo:quadratic-space-query}
    \begin{algorithmic}[1] % The number tells where the line numbering should start
        \Procedure{edge-cotained}{$s,t,x,y$}
            \State{${\mu}\gets$ LCA($\mathcal T_G,s,t$)}
            \State $\mathcal P_1 \gets P(\pi_{S(\mu)}(s),\pi_{S(\mu)}(t))$
            \State $\mathcal P_2 \gets P(x,y)$
            \If{$\mathcal P_1 \cap \mathcal P_2 = \varnothing$} 
            \State \textbf{return} False
            \Else 
            \State \textbf{ return} True
            \EndIf
        \EndProcedure
    \end{algorithmic}
\end{algorithm}
We can thus state the following theorem.
\begin{theorem}
 Given an undirected graph $G=(V,E)$ on $n=|V|$ vertices, there exists a data structure of 
${\cal O}(n^2)$ size that takes ${\cal O}(1)$ time to determine whether an edge $(x,y)\in E$
belongs to a $(s,t)$-mincut for any $s,t\in V$
and $(x,y)\in E$.
% \label{thm:O(n^2)-size-data-structure}
\end{theorem}


\section{Size and Time analysis of compact data structure}
\label{appendix:size-time-analysis-compact-ds}

The data structure doesn't seem to be an ${\cal O}(m)$ size data structure at first sight. Observe that augmentation at any internal node can still take ${\cal O}(m)$ space individually. Interestingly, we show that collective space taken by augmentation at each internal node will still be ${\cal O}(m)$.

We begin with the following lemma which gives a tight bound on the sum of weights of edges in Gomory-Hu tree is $\Theta(m)$. We also give the proof for the same which was suggested in \cite{DBLP:conf/stoc/HariharanKPB07,DBLP:journals/siamcomp/DinitzV00}.

\begin{lemma}[\cite{DBLP:conf/stoc/HariharanKPB07,DBLP:journals/siamcomp/DinitzV00}]
\label{fact:GH-weight}
The sum of weights of all edges in the Gomory-Hu tree is ${\Theta}(m)$.
\end{lemma}
\begin{proof}
Consider any edge $(u,v)\in E$. This edge must be present in every $(u,v)$-mincut. Thus, the sum of weights of all edges in Gomory-Hu tree is at least $m$. Now, root the Gomory-Hu tree at any arbitrary vertex $r$. Let $f$ maps each edge in this tree to its lower end-point. It is easy to observe that $f$ is a one-to-one mapping. Let $e$ be an edge in the Gomory-Hu tree. Observe that $w(e) \leq deg(f(e))$ (where $deg()$ is the degree of vertex in $G$). Thus the sum of weights of all edges in Gomory-Hu tree is at most $2m$. This comes from the simple observation that the sum of the degree of all vertices in $G$ equals $2m$.
\end{proof}

Let us assign each edge $(\mu,\mu')$ in the hierarchy tree ${\cal T}_G$ weight equal to the Steiner mincut value for the Steiner set $S(\mu)$ (if $\mu$ is the parent of $\mu'$). We shall show that sum of the weight of edges in hierarchy tree ${\cal T}_G$ is $\Theta(m)$. %This will help in our analysis later.

To establish this bound refer to algorithm \ref{Construct Tree} that gives an algorithm to construct the hierarchy tree from Gomory-Hu tree. Observe that the variable $ctr$ in this algorithm stores the sum of weights of all edges in ${\cal T}_G$. It is clear that for $k$ edges removed from the Gomory-Hu tree, we add $k+1$ edges of equal weight in ${\cal T}_G$. Thus, the sum of the weight of all edges in ${\cal T}_G$ is at most $4m$ (since $k+1 \leq 2k$). Therefore, we state the following lemma.

\begin{algorithm}%[H]
    \caption{Construct Hierarchy Tree ${\cal T}_G$ from Gomory-Hu Tree $\hat{ \cal T}_{G}$}
    \label{Construct Tree}
    \begin{algorithmic}[1] % The number tells where the line numbering should start
        \State $ctr \gets 0$
        \Procedure{Construct-Tree}{$\hat{ \cal T}_{G}$}
            \If{$\hat{ \cal T}_{G}$ has single node}
            \State Create a node $\mu$
            \State $val(\mu) \gets val(\hat{ \cal T}_{G})$
            \State \textbf{return} $\mu$
            \EndIf
            \State $c_{\min} \gets$ $\min_{e\in \hat{ \cal T}_{G}}w(e)$
            \State Let there be $k$ edges with weight $c_{\min}$
            \State Remove all edges of weight $c_{\min}$ in $\hat{ \cal T}_{G}$ to get $(k+1)$ trees $T_1,..,T_{k+1}$
            \State Create a node $\mu$
            \State $ctr \gets ctr + c_{\min}\times(k+1)$
            \State $children(\mu) \gets \{\textsc{Construct-Tree}(T_i)\;|\;\forall i\in [k+1]\}$
            \State \textbf{return} $\mu$
        \EndProcedure
    \end{algorithmic}
\end{algorithm}

\begin{lemma}
\label{lem:hierarchy-tree-weight}
The sum of weights of all edges in the tree ${\cal T}_G$ is ${\Theta}(m)$.
\end{lemma}
% The size and time analysis of the data structure crucially exploits the following observations,
We shall now give a bound on the size of connectivity carcass augmented at each internal node. The following lemma gives a bound on the size of flesh graph ${\cal F}_S$ for any Steiner set $S$.

\begin{lemma}
Let ${\cal V}_S$ and ${\cal W}_S$ denote the set of Steiner and non-Steiner units respectively in flesh graph ${\cal F}_S$ with Steiner set $S\subseteq V$. The size of ${\cal F}_S$ is bounded by $|{\cal V}_S|c_S + \sum_{u\in {\cal W}_S}deg(u)$.
\label{lem:size-of-flesh}
\end{lemma}
\begin{proof}
Consider the Gomory-Hu tree of the flesh ${\cal F}_S$, say ${\cal T}$. It is evident that the value of mincut between any two units is at most $c_S$. This follows from the definition of a unit. Now, root this tree ${\cal T}$ at some Steiner unit. Let $f$ maps each edge in this tree to its lower end-point. Any edge in this tree has weight at most $c_S$. However, for any non-Steiner unit $u$, $w(f^{-1}(u)) \leq deg(u)$ (where $deg()$ is the degree of vertex in $G$).  Thus, the sum of weight of all edges in ${\cal T}$ is bounded by $|{\cal V}_S|c_S + \sum_{u\in {\cal W}_S}deg(u)$. Using Lemma \ref{fact:GH-weight}, it follows that size of flesh ${\cal F}_S$ is bounded by $|{\cal V}_S|c_S + \sum_{u\in {\cal W}_S}deg(u)$.
\end{proof}

Consider the flesh graph ${\cal F}_{S(\mu)}$ stored at some internal node $\mu$ in the tree. Let ${\cal V}_{S(\mu)}$ and ${\cal W}_{S(\mu)}$ denote the set of Steiner and non-Steiner units respectively in ${\cal F}_{S(\mu)}$. Let $u$ be some non-Steiner unit in ${\cal F}_{S(\mu)}$. It is evident that $u$ consists of only contracted vertices (obtained after the contraction procedure at some ancestral node). This non-Steiner unit gets compressed to a new contracted vertex in all descendants, and in a sense, disappears. Thus, each contracted vertex appears in at most one non-Steiner unit. 

Now, we shall count the total number of contracted vertices introduced at each internal node. We know that this count is an upper bound on the total number of non-Steiner units across all flesh graphs. It follows from Lemma \ref{lem:contracted-subcactus-mincut} that the number of contracted vertices introduced by node $\mu$ to the graph $G_{\mu'}$ associated with its child $\mu'$ is equal to the number of cycles and tree edges incident on node corresponding to $\mu'$ in skeleton ${\cal H}_{S(\mu)}$. We sum this number for each child of $\mu$. The total number of contracted vertices introduced by internal node $\mu$ to all its children is at most twice the number of tree and cycle edges. Since the skeleton is a cactus graph, thus the number of tree and cycle edges is ${\cal O}(|{\cal V}_{S(\mu)}|)$. Moreover, we know that the number of Steiner units in ${\cal F}_{S(\mu)}$ also equals the number of children of node $\mu$ in tree ${\cal T}_G$. Therefore, the number of Steiner units across all flesh graphs stored at each internal node is given by ${\sum}_{\mu \in {\cal T}_G}|{\cal V}_{S(\mu)}|$ which is ${\cal O}(n)$. Thus, the total number of non-Steiner units across all flesh graphs is also ${\cal O}(n)$.

Now, we shall bound the sum of the degree of all non-Steiner units across all flesh graphs. Since each contracted vertex appears in at most one non-Steiner unit, we can sum the degree of all contracted vertices to get an upper bound. It again follows from Lemma \ref{lem:contracted-subcactus-mincut} that the degree of contracted vertex introduced by node $\mu$ is exactly $c_{S(\mu)}$. Thus, the sum of degree of all contracted vertices introduced by node $\mu$ is ${\cal O}(|{\cal V}_{S(\mu)}|c_{S(\mu)})$.

Combining the above observations, we can infer the following.


\begin{inference}
\label{inf:units-O(n)}
The total number of units across all flesh graphs is ${\cal O}(n)$.
\end{inference}
\begin{inference}
\label{inf:sum-degree}
The sum of degree of non-Steiner units across all flesh graphs stored at each internal node $\mu$ is ${\cal O}(\sum_{\mu \in {\cal T}_G}{|{\cal V}_{S(\mu)}|}c_{S(\mu)})$ i.e ${\cal O}(m)$ (follows from Lemma \ref{lem:hierarchy-tree-weight}). In other words, $\sum_{\mu \in {\cal T}_G} \sum_{u \in {\cal W}_{S(\mu)}}deg(u)$ is ${\cal O}(m)$.
\end{inference}


\subsubsection*{Size analysis of Data Structure}
Combining Lemma \ref{fact:GH-weight}, Lemma \ref{lem:hierarchy-tree-weight}, Lemma \ref{lem:size-of-flesh} and Inference \ref{inf:sum-degree} we get the following result.
\begin{equation*}
    \begin{split}
        \nonumber
        \sum_{\mu \in {\cal T}_G}|{\cal F}_{S(\mu)}| &\leq c_1 \times \sum_{\mu \in {\cal T}_G} (|{\cal V}_{S(\mu)}|c_{S(\mu)} + \sum_{u\in {\cal W}_{S(\mu)}}deg(u))\\
        &\leq c_2 \times m
    \end{split}
\end{equation*}


\subsubsection*{Time analysis of Data Structure}

A trivial bound on the query time follows from the size analysis itself. Since the combined size of our data structure is ${\cal O}(m)$, it follows that the sum of sizes of all flesh graphs from the root node to $LCA(s,t)$ will also be ${\cal O}(m)$. Dinitz and Vainshtein \cite{DBLP:conf/stoc/DinitzV94} showed that size of flesh graph ${\cal F}_S$ is ${\cal O}(\tilde{n}c_S)$ for Steiner set $S$, where $\tilde{n}$ is the number of units in the flesh graph. Total number of units across all flesh graphs is only ${\cal O}(n)$ (Inference \ref{inf:units-O(n)}). The value of Steiner mincut increases as we traverse from the root towards a leaf. Thus, $c_{s,t}$ is the maximum Steiner mincut value in the path from root node to $LCA(s,t)$ . Thus, the sum of sizes of all flesh graphs in this path is bounded by ${\cal O}(nc_{s,t})$. Thus, the query time we achieve is ${\cal O}(\min(m,nc_{s,t}))$.